\documentclass[fleqn,8pt,t]{beamer}

\usepackage[english]{babel}
\usepackage[utf8]{inputenc}
\usepackage[T1]{fontenc}
%\usepackage{french} % Sommaire en début de document
%\usepackage[top=2cm, bottom=2cm, left=2cm, right=2cm]{geometry} % Marges

\usepackage{amsmath} % Maths
\usepackage{amsfonts} % Maths
\usepackage{amssymb} % Maths
\usepackage{stmaryrd} % Maths (crochets doubles)

%\usepackage{listings} % Mise en forme du code (pour Hoare) ## À REVOIR ###
%\usepackage{ifthen} % Structures If Then Else
\usepackage{theorem} % Styles supplémentaires pour théorèmes
\usepackage{url}
\usepackage{array}  % Tableaux évolués
\usepackage{multirow}  % Pour des colonnes sur plusieurs lignes

%\usepackage{enumerate} % Changer les puces des listes d'énumération
%\usepackage{setspace} % Changer les interlignes

%\usepackage{subfig} % Créer des sous-figures
%\usepackage{graphicx} % Importer des images
\usepackage{tikz}

\usepackage{ulem}  % Pour l'attribut barré

\usepackage{comment}

% Police
\usepackage{lmodern}
%\usepackage{libertine}


\input{macros/macros}
\input{macros/macros-ph}
\input{macros/macros-abstr}
\input{macros/tikzstyles}



% Commande À FAIRE
\usepackage{color} % Couleurs du texte
%\newcommand{\afaire}[1]{\textcolor{red}{[À FAIRE : #1]}}
\newcommand{\todo}[1]{\textcolor{red}{<[[#1]]>}}



\colorlet{couleurtheme}{gray}  % Couleur principale du thème
\colorlet{couleurcit}{gray}  % Couleur des citations
\colorlet{couleurex}{blue}  % Couleur des citations
\colorlet{couleurliens}{darkblue}  % Couleur des citations

\usetheme{Pittsburgh}   % Thème général
\usefonttheme{default}  % Thème de polices
\setbeamertemplate{navigation symbols}{}  % Pas de menu de navigation
%\setbeamertemplate{itemize item}[x]   % Puces des listes

\usecolortheme[named=couleurtheme]{structure}    % Couleur de la structure : titres et puces
%\setbeamercolor{normal text}{bg=black,fg=white}  % Couleur du texte
\setbeamercolor{item}{fg=couleurtheme}           % Couleur des puces
%\setbeamercolor{item projected}{fg=black}        % Couleur des recouvrements
%\setbeamercolor{alerted text}{fg=yellow}         % ?

\setbeamerfont{frametitle}{size=\Large}  % Police des titres


% Flèche grise
\newcommand{\f}{\textcolor{couleurtheme}{\textbf{$\rightarrow$\ }}}
\newcommand{\cth}[1]{\textcolor{couleurtheme}{#1}}

% Environnement liste avec flèches
\newenvironment{fleches}{%
\begin{list}{}{%
\setlength{\labelwidth}{1em}% largeur de la boîte englobant le label
\setlength{\labelsep}{0pt}% espace entre paragraphe et l’étiquette
%\setlength{\itemsep}{1pt}
%\setlength{\leftmargin}{\labelwidth+\labelsep}% marge de gauche
\renewcommand{\makelabel}{\f}%
}}{\end{list}}

% Liste sans puce
\newenvironment{liste}{%
\begin{list}{}{%
\setlength{\labelwidth}{0em}% largeur de la boîte englobant le label
\setlength{\labelsep}{0pt}% espace entre paragraphe et l’étiquette
\setlength{\leftmargin}{0em}% marge de gauche
%\renewcommand{\makelabel}{\f}%
}}{\end{list}}

% Style des exemples
\newcommand{\ex}[1]{\textcolor{couleurex}{#1}}
\newcommand{\qex}[1]{\quad \ex{#1}}
\newcommand{\rex}[1]{\hfill \ex{#1}}
\newcommand{\redex}[1]{\textcolor{red}{#1}}

\newcommand{\lien}[1]{\textcolor{couleurliens}{\underline{\url{#1}}}}

\newcommand{\console}[1]{\textcolor{darkgray}{#1}}

% Style des citations
\newcommand{\tscite}[1]{\textcolor{couleurcit}{#1}}
\newcommand{\tcite}[1]{\textcolor{couleurcit}{[#1]}}
\newcommand{\tcitebullet}{~~$\textcolor{couleurtheme}{\bullet}$~}



% Style de texte mis en valeur
\newcommand{\tval}[1]{\textbf{#1}}

% Un vrai symbole pour l'ensemble vide
\renewcommand{\emptyset}{\varnothing}

% Pour définir la conférence et son nom court
\newcommand{\conference}[2]{\def\theconference{#2}
\def\insertshortconference{\ifthenelse{\equal{#1}{-}}{#2}{\ifthenelse{\equal{#1}{}}{#2}{#1}}}}



\newcommand{\thedate}{2014/06/03}
\date{\thedate}
\conference{BioTempo}{Réunion finale BioTempo}
\title[Introducing time into PH with priorities]{Introducing time parameters into Process Hitting with classes of priorities}
\author{Maxime FOLSCHETTE}




\setbeamertemplate{footline}{\color{gray}%
\scriptsize
\quad\strut%
\insertauthor%
\hfill%
\insertframenumber/\inserttotalframenumber%
\hfill%
\insertshortconference{} --- \thedate\quad\strut
}


\newcommand{\headersep}{$\circ$} % \bullet \triangleright

\setbeamertemplate{headline}{\color{gray}%
\vskip0.3em%
\quad\strut%
{\scriptsize\color{black}%
% Gris si une section existe
\ifthenelse{\equal{\thesection}{0}}{}{%
\ifthenelse{\equal{\lastsection}{x}}{}{%
\color{gray}%
}}%
\insertshorttitle
\ifthenelse{\equal{\thesection}{0}}{}{%
\ifthenelse{\equal{\lastsection}{x}}{}{%
~\headersep{} %
% Gris si une sous-section existe
\ifthenelse{\equal{\thesubsection}{0}}{\color{black}}{%
\ifthenelse{\equal{\lastsubsection}{x}}{\color{black}}{%
\color{gray}%
}}%
\insertsectionhead%
%
\ifthenelse{\equal{\thesubsection}{0}}{}{%
\ifthenelse{\equal{\lastsubsection}{x}}{}{%
~\headersep{} \color{black}\insertsubsectionhead%
%
}}}}}%
\vskip-5ex%
}



\def \scaleex {0.85}
\def \scaleminiex {0.6}
\def \scaleinf {0.6}

\colorlet{colorb}{blue}
\colorlet{colora1}{yellow}
\colorlet{colora0}{green}
\colorlet{colora1font}{darkyellow}
\colorlet{colora0font}{darkgreen}

\colorlet{exanswer}{blue}
\colorlet{colorgray}{lightgray}

\definecolor{colortitle}{rgb}{0.54,0.8,0.9}


\begin{document}

\begin{frame}[plain,label=title]

% Cadre de titre
\begin{center}
\vspace{1cm}
\setbeamercolor{postit}{fg=black,bg=colortitle}
\begin{beamercolorbox}[sep=0.5em]{postit}
\centering
\Large
\textbf{%
{\normalsize\theconference{}}\\~\\%
\inserttitle
}
\end{beamercolorbox}

% Auteurs et instituts
\par
\medskip
\bigskip
\normalsize
Maxime FOLSCHETTE

\medskip
\footnotesize
MeForBio / IRCCyN / École Centrale de Nantes (Nantes, France)

\texttt{maxime.folschette@irccyn.ec-nantes.fr}

\url{http://www.irccyn.ec-nantes.fr/~folschet/}

\bigskip
Joint work with:
\\
\normalsize
Loïc PAULEVÉ, Morgan MAGNIN, Olivier ROUX
\end{center}

\end{frame}



\input{parts/ex.tex}

% Références des modèles
\newcommand{\cmodels}{\bigskip
\quad\tval{\ex{egfr20}}: \tcite{Epidermal Growth Factor Receptor, by Özgür Sahin \textit{et al.}}\\
\quad\tval{\ex{egfr104}}: \tcite{Epidermal Growth Factor Receptor, by Regina Samaga \textit{et al.}}\\
\quad\tval{\ex{tcrsig40}}: \tcite{T-Cell Receptor Signaling, by Steffen Klamt \textit{et al.}}\\
\quad\tval{\ex{tcrsig94}}: \tcite{T-Cell Receptor Signaling, by Julio Saez-Rodriguez \textit{et al.}}\\}

% Under-approximation of Reachability in Multivalued Asynchronous Networks
\newcommand{\cfpmrcsbio}{Folschette \textit{et al.} in Workshop on Interactions between Computer Science and Biology, 2013}
% Refining dynamics of gene regulatory networks in a stochastic $\pi$-calculus framework
\newcommand{\cpmrtcsb}{Paulevé \textit{et al.} in Transactions on Computational Systems Biology, 2011}
% Static analysis of biological regulatory networks dynamics using abstract interpretation
\newcommand{\cpmrmscs}{Paulevé \textit{et al.} in Mathematical Structures in Computer Science, 2012}
% Concretizing the Process Hitting into Biological Regulatory Networks
%\newcommand{\cfpimrcmsb}{Folschette \textit{et al.} in Computational Methods in Systems Biology, 2012}
% Formal Methods for Modeling Biological Regulatory Networks
%\newcommand{\crcbmfma}{Richard \textit{et al.} in Modern Formal Methods and App., 2006}
% Semantics of Biological Regulatory Networks
%\newcommand{\bccdmr}{Bernot \textit{et al.} in Concurrent Models in Molecular Biology, 2007}
% Negative circuits and sustained oscillations in asynchronous automata networks
%\newcommand{\rcite}{Richard in Advances in Applied Mathematics, 2010}
% Thèse de Loïc
\newcommand{\paulevephd}{Paulevé (PhD thesis), 2011}



\section{Introduction}
% Diapo d'intro

\begin{frame}[c]
  \frametitle{Context and Aims}

\tval{MeForBio} team: Algebraic modeling to study large dynamical biological systems

\bigskip
\f Contribution: the \tval{Process Hitting} framework

{\small\tcite{\cpmrtcsb}\\
\tcite{\cpmrmscs}}

\begin{itemize}
  \item A restriction of synchronous automata networks
  \item Special form for the actions $\Rightarrow$ more atomistic than Interaction Graphs
  \item Efficient reachability analysis
\end{itemize}


\bigskip
\f Introduction of temporal features:

\begin{enumerate}[1)]
  \item Stochastic parameters\\
{\small\tcite{\cpmrtcsb}}
  \item Priorities\\
{\small\tcite{\cfpmrcsbio}}
  \item Neutralizing edges
\end{enumerate}

\end{frame}


\section{The Process Hitting framework}
\subsection{Definition}
% Définition du Process Hitting + sortes coopératives

\begin{frame}[c]
  \frametitle{Les Frappes de Processus standards}

Les \tval{Frappes de Processus standards} :
\begin{itemize}
  \item Adaptées à la représentation des RRB
  \item Modélisation \tval{atomique et qualitative} (niveaux discrets explicites)
  \item Dynamique \tval{simple mais puissante} (forme des actions contrainte)
\end{itemize}

\pause
\bigskip
Outils développés précédemment :
\begin{itemize}
  \item \tval{Analyse d'atteignabilité}
  \item Recherche de points fixes
  \item Paramètres stochastiques
\end{itemize}

\medskip
\f Formalisme bien adapté à l'étude des \tval{grands réseaux de régulation}

\pause
\bigskip
Quelques lacunes :
\begin{itemize}
  \item Représentation inexacte des \tval{coopérations}
  \item \tval{Enrichissement possible} de l'expressivité\\
    \quad \f Nécessité d'adapter les outils développés
\end{itemize}

\end{frame}



\begin{frame}[t]
  \frametitle{Les Frappes de Processus standards}
  \framesubtitle{\tcite{\cpmrtcsb}}

% 1 : Sortes
\only<1>{
\tikzstyle{process}=[circle,minimum size=15pt,font=\footnotesize,inner sep=1pt]
\tikzstyle{tick label}=[color=white, font=\footnotesize]
\tikzstyle{tick}=[transparent]
\tikzstyle{hit}=[transparent]
\tikzstyle{selfhit}=[transparent, min distance=30pt,curve to]
\tikzstyle{bounce}=[transparent]
\tikzstyle{hlhit}=[transparent]
\begin{center}\scalebox{\scaleex}{
\begin{tikzpicture}
  \exphdef
\end{tikzpicture}
}\end{center}
}

% 2 : Processus
\only<2>{
\tikzstyle{process}=[circle,draw,minimum size=15pt,font=\footnotesize,inner sep=1pt]
\tikzstyle{tick label}=[font=\footnotesize]
\tikzstyle{tick}=[densely dotted]
\tikzstyle{hit}=[transparent]
\tikzstyle{selfhit}=[transparent, min distance=30pt,curve to]
\tikzstyle{bounce}=[transparent]
\tikzstyle{hlhit}=[transparent]
\begin{center}\scalebox{\scaleex}{
\begin{tikzpicture}
  \exphdef
\end{tikzpicture}
}\end{center}
}

% 3 : États
\only<3>{
\tikzstyle{hit}=[transparent]
\tikzstyle{selfhit}=[transparent, min distance=30pt,curve to]
\tikzstyle{bounce}=[transparent]
\tikzstyle{hlhit}=[transparent]
\begin{center}\scalebox{\scaleex}{
\begin{tikzpicture}
  \exphdef

  \TState{3}{a_0,b_1,z_0}
\end{tikzpicture}
}\end{center}
}

% 4 : Actions
\only<4->{
\tikzstyle{tick}=[densely dotted]
\tikzstyle{hit}=[->,>=angle 45]
\tikzstyle{selfhit}=[min distance=30pt,curve to]
\tikzstyle{bounce}=[densely dotted,>=stealth',->]
\tikzstyle{hlhit}=[very thick]
\begin{center}\scalebox{\scaleex}{
\begin{tikzpicture}
\exphdef
  \TState{4-5}{a_0,b_1,z_0}
  \TState{6}{a_0,b_1,z_1}
  \TState{7}{a_1,b_1,z_1}
  \TState{8}{a_1,b_1,z_2}

  \only<5>{
    \THit{b_1}{hl}{z_0}{.west}{z_1}
    \path[bounce,bend left,hl] \TBounce{z_0}{}{z_1}{.south};
  }
  \only<6>{
    \THit{a_0}{out=250,in=200,selfhit,hl}{a_0}{.west}{a_1}
    \path[bounce,bend left,hl] \TBounce{a_0}{}{a_1}{.south};
  }
  \only<7>{
    \THit{a_1}{hl}{z_1}{.west}{z_2}
    \path[bounce,bend left,hl] \TBounce{z_1}{}{z_2}{.south};
  }
\end{tikzpicture}
}\end{center}
}

\medskip
\begin{liste}
  \item \tval{Sortes} : composants \qex{$a$, $b$, $z$}
\pause[2]
  \item \tval{Processus} : états locaux / niveaux d'expression \qex{$z_0$, $z_1$, $z_2$}
\pause[3]
  \item \tval{États} : ensembles de processus actifs%
  \only<3-5>{\qex{$\PHetat{a_0, b_1, z_0}$}}%
  \only<6>{\qex{$\PHetat{a_0, b_1, z_1}$}}%
  \only<7>{\qex{$\PHetat{a_1, b_1, z_1}$}}%
  \only<8>{\qex{$\PHetat{a_1, b_1, z_2}$}}%
\pause[4]
  \item \tval{Actions} : dynamique \qex{\only<5>{\underline}{$\PHfrappe{b_1}{z_0}{z_1}$}, \only<6>{\underline}{$\PHfrappe{a_0}{a_0}{a_1}$}, \only<7>{\underline}{$\PHfrappe{a_1}{z_1}{z_2}$}}
\end{liste}
\end{frame}



\begin{frame}
  \frametitle{Coopérations}
  \framesubtitle{\tcite{\cpmrtcsb}}

\begin{center}\scalebox{\scaleex}{
\begin{tikzpicture}
  \exphcoop
  
  \TState{9}{a_1,b_1}
  \TState{10}{a_1,b_1,ab_0}
  \TState{11}{a_1,b_1,ab_1}
  \TState{12}{a_1,b_1,ab_3}
  
  \node at (a_1.center) {\textasteriskcentered};
  \node at (b_1.center) {\textasteriskcentered};
  \node<5-> at (ab_3.center) {\textasteriskcentered};
  
  \only<9>{
    \node[hl process] at (ab_0.center) {};
    \node[hl process] at (ab_1.center) {};
    \node[hl process] at (ab_2.center) {};
    \node[current process] at (ab_3.center) {};
  }
  
  \only<10>{
    \THit{b_1}{hl}{ab_0}{.210}{ab_1}
    \path[bounce,bend left,hl] \TBounce{ab_0}{}{ab_1}{.240} ;
  }
  
  \only<11>{
    \THit{a_1}{hl}{ab_1}{.160}{ab_3}
    \path[bounce,bend left,hl] \TBounce{ab_1}{}{ab_3}{.south west} ;
  }
\end{tikzpicture}
}\end{center}

\medskip
\begin{liste}
  \item \tval{Coopération} entre \ex{$a_1$} et \ex{$b_1$} : \qex{$\PHfrappe{\underline{a_1 \wedge b_1}}{z_0}{z_1}$}
\pause[5]
  \item Solution : une \tval{sorte coopérative} \qex{$ab$} \quad pour exprimer \qex{$\underline{a_1 \wedge b_1}$}
\pause[9]
  \item Chaque configuration est représentée par un processus \qex{$\underline{a_1 \wedge b_1} \Rightarrow ab_{11}$}
%\pause[15]
%  \item Advantage: regular sort; drawbacks: complexity, temporal shift
\end{liste}
\end{frame}

\subsection{Static Analysis}
\input{parts/annex_static.tex}
\subsection[x]{Summary}
\input{parts/ccl_ph.tex}

\section{The Metazoan Segmentation Model}
\input{parts/metazoan.tex}

\section{Stochastic Features}
\subsection[x]{Stochastic Features in Process Hitting}
\input{parts/def_stocha.tex}
\subsection{Adding Stochasticity to the Metazoan Segmentation}
\input{parts/metazoan_stocha.tex}
\subsection[x]{Summary}
\input{parts/ccl_stocha.tex}

\section{Priorities}
\subsection[x]{Classes of Priorities in Process Hitting}
% Définition des priorités

\begin{frame}[t]
  \frametitle{Introduction de classes de priorités}
  \framesubtitle{\tcite{\cfpmrcsbio}}

\bigskip
\begin{itemize}
  \item À chaque action est associée une classe de priorité
  \item Une action n'est jouable que si aucune action plus prioritaire ne l'est
\end{itemize}

\medskip

\begin{center}
% \begin{tabular}{ccccc}
%   \hspace*{.3cm}\tikz \node[labelprio1] {$1$}; \hspace*{.3cm} &
%   \hspace*{.3cm}\tikz \node[labelprio2] {$2$}; \hspace*{.3cm} &
%   \hspace*{.3cm}\tikz \node[labelprio3] {$3$}; \hspace*{.3cm} &
%   \vspace*{.5em}\hspace*{.3cm}\raisebox{5pt}{\ldots}\hspace*{.3cm} &
%   \hspace*{.3cm}\tikz \node[labelprion] {$n$}; \hspace*{.3cm} \\\hline
%   \multicolumn{2}{l}{
%   \parbox{1.5cm}{\vspace*{.5em}plus haute\\priorité}} &
%   %\parbox{1cm}{~} &
%   \parbox{1cm}{~} &&
%   \parbox{1.5cm}{\vspace*{.5em}plus basse\\priorité}
% \end{tabular}
% \hspace*{-1em}
% \raisebox{2.2pt}{$\blacktriangleright$}

\begin{tabular}{*{5}{>{\centering}p{1cm}}}
  \tikz \node[labelprio1] {$1$}; &
  \tikz \node[labelprio2] {$2$}; &
  \tikz \node[labelprio3] {$3$}; &
  \raisebox{5pt}{\ldots} &
  \tikz \node[labelprion] {$n$};
\vspace*{.5em} \tabularnewline \hline
  \multicolumn{2}{l}{\parbox{1.5cm}{\vspace*{.5em}le plus\\prioritaire}} &&
  \multicolumn{2}{r}{\parbox{1.5cm}{\raggedleft\vspace*{.5em}le moins\\prioritaire}}
\end{tabular}
\hspace*{-1em}
\raisebox{2.2pt}{$\blacktriangleright$}

\bigskip

\only<2>{
\bigskip
\begin{tikzpicture}
  \path[use as bounding box] (-0.5,-0.5) rectangle (2.5,1.5);
  \TSort{(0,0)}{a}{2}{l}
  \TSort{(2,0)}{b}{2}{r}
  \THit{a_0}{}{b_0}{.west}{b_1}
  \THit{a_0}{out=-120,in=180,selfhit}{a_0}{.west}{a_1}
  \path[bounce]
  \TBounce{a_0}{bend left}{a_1}{.south}
  \TBounce{b_0}{bend left}{b_1}{.south}
  ;
  \TState{-2}{a_0,b_0}
  \TState{3-}{a_1,b_0}

  \node[labelprio1] at (-1.5,-0.5) {$1$};
  \node[labelprio2] at (1,0.25) {$2$};
\end{tikzpicture}

\bigskip

\f $b_1$ n'est \tval{jamais atteignable}
}
\end{center}

\only<3->{
\begin{itemize}
  \item Permet de modéliser des classes d'actions de vitesses similaires
\end{itemize}
\begin{center}
\begin{tabular}{*{5}{>{\centering}p{1cm}}}
  \tikz \node[labelprio1,labelstocha] {$A$}; &
  \tikz \node[labelprio2,labelstocha] {$B$}; &
  \tikz \node[labelprio3,labelstocha] {$C$}; &
  \raisebox{5pt}{\ldots} &
  \tikz \node[labelprion,labelstocha] {$N$};
\vspace*{.5em} \tabularnewline \hline
  \multicolumn{1}{r}{\parbox{1cm}{\hspace*{-1.7cm}\parbox{2.5cm}{\raggedleft\vspace*{.5em}\tval{instantanée}\\(non contrôlable)}}} &
  \multicolumn{2}{l}{\parbox{2cm}{\vspace*{.5em}\tval{très rapide}\\(contrôlable)}} &
  \multicolumn{2}{r}{\parbox{2cm}{\raggedleft\vspace*{.5em}\tval{très lente}\\~}}
\end{tabular}
\hspace*{-1em}
\raisebox{2.2pt}{$\blacktriangleright$}
\end{center}
}

\end{frame}

\subsection{Adding Priorities to the Metazoan Segmentation}
% Priorités dans Metazoan

\begin{frame}[t]
  \frametitle{Metazoan Segmentation with Priorities}
  \framesubtitle{\tcite{\cfpmrcsbio}}

\makenoprio

\begin{tikzpicture}
  \path[use as bounding box] (-2,0) rectangle (8,6.5);
  \exmetazoan

  \node[labelprio1] at (2,3.3) {$1$};
  \node[labelprio1] at (2.3,1.6) {$1$};
  \node[labelprio2] at (5.5,3.9) {$2$};
  \node[labelprio2] at (3.5,5.3) {$2$};
  \node[labelprio3] at (0,2.5) {$3$};
  \node[labelprio3] at (0.8,5.8) {$3$};
  
  \TState{3,9,15}{f_1, a_0, c_0, fc_2}
  \TState{4,10}{f_1, a_1, c_0, fc_2}
  \TState{5,11}{f_1, a_1, c_1, fc_2}
  \TState{6,12}{f_1, a_1, c_1, fc_3}
  \TState{7,13}{f_1, a_0, c_1, fc_3}
  \TState{8,14}{f_1, a_0, c_0, fc_3}
\end{tikzpicture}

\pause[2]
\vspace*{-2.5cm}
\hfill
\begin{tikzpicture}
  \tikz \foreach \x in {0,...,12}
    \draw[dotted] (\x/4,0) -- (\x/4,1.5);

  \draw[dotted] (0,0) -- (-3,0);
  \draw[dotted] (0,1.5) -- (-3,1.5);

  \only<4->{\fill (-3,0) rectangle (-2.75,1.5);}
  \only<5->{\fill (-2.75,0) rectangle (-2.5,1.5);}
  \only<6->{\fill (-2.5,0) rectangle (-2.25,1.5);}
  \only<7->{\fill[gray!30] (-2.25,0) rectangle (-2,1.5);}
  \only<8->{\fill[gray!30] (-2,0) rectangle (-1.75,1.5);}
  \only<9->{\fill[gray!30] (-1.75,0) rectangle (-1.5,1.5);}
  \only<10->{\fill (-1.5,0) rectangle (-1.25,1.5);}
  \only<11->{\fill (-1.25,0) rectangle (-1,1.5);}
  \only<12->{\fill (-1,0) rectangle (-0.75,1.5);}
  \only<13->{\fill[gray!30] (-0.75,0) rectangle (-0.5,1.5);}
  \only<14->{\fill[gray!30] (-0.5,0) rectangle (-0.25,1.5);}
  \only<15->{\fill[gray!30] (-0.25,0) rectangle (0,1.5);}
\end{tikzpicture}

\pause[15]
%\vspace*{.3cm}
\begin{flushright}
  \f Only one\\possible behavior
\end{flushright}


\end{frame}



\begin{frame}[c]
  \frametitle{Metazoan Segmentation in Canonical Form}

\makenoprio

\vspace*{.5cm}
\scalebox{.9}{
\begin{tikzpicture}
  \path[use as bounding box] (-5.75,0) rectangle (5.75,5.5);
  \TSort{(-5,4)}{c}{2}{l}
  \TSort{(0,1)}{f}{2}{l}
  \TSort{(5,4)}{a}{2}{r}

  \TSetTick{fc}{0}{00}
  \TSetTick{fc}{1}{01}
  \TSetTick{fc}{2}{10}
  \TSetTick{fc}{3}{11}
  \TSort{(3,0)}{fc}{4}{r}
  
  \THit{fc_2}{}{a_0}{.south west}{a_1}
  \path[bounce, bend left=60]
    \TBounce{a_0}{}{a_1}{.south west};
  
  \THit{c_1.north east}{}{a_1}{.west}{a_0}
  \path[bounce, bend right=50]
    \TBounce{a_1}{}{a_0}{.north west};
  
  \path (0.8, 1.5) edge[\prio,coopupdate] (2.2, 1.5);
  \path (-4.3, 4.5) edge[\prio,coopupdate] (2.2, 2.5);
  
  \only<1>{
    \THit{c_1.north}{selfhit}{c_1}{.west}{c_0}
    \path[bounce, bend right=50]
      \TBounce{c_1}{}{c_0}{.north west};
  }
  
  \only<2>{
    \THit{c_1.north}{selfhit,hlr}{c_1}{.west}{c_0}
    \path[bounce, bend right=50]
      \TBounce{c_1}{hlr}{c_0}{.north west};
  }
  
  \only<-2>{
    \node[labelprio3] at (-4.4,6) {$3$};
  }
  
  \only<3>{
    \THit{a_0.west}{hlv}{c_1}{.east}{c_0}
    \path[bounce, bend left=50]
      \TBounce{c_1}{hlv}{c_0}{.north east};
    \node[labelprio2] at (0.5,4.7) {$2$};
  }
  
  \only<4->{
    \THit{a_0.west}{}{c_1}{.east}{c_0}
    \path[bounce, bend left=50]
      \TBounce{c_1}{}{c_0}{.north east};
    \node[labelprio2] at (0.5,4.7) {$2$};
  }
  
  \only<-4>{
    \node[labelprio3] at (-3,2.3) {$3$};
  }
  
  \only<-3>{
    \THit{f_1}{bend left=30, in=90}{c_0}{.west}{c_1}
    \path[bounce, bend left=50]
      \TBounce{c_0}{}{c_1}{.south west};
  }
  
  \only<4>{
    \THit{f_1}{bend left=30, in=90, hlr}{c_0}{.west}{c_1}
    \path[bounce, bend left=50]
      \TBounce{c_0}{hlr}{c_1}{.south west};
  }
  
  \only<5->{
    \TSetTick{fa}{0}{00}
    \TSetTick{fa}{1}{01}
    \TSetTick{fa}{2}{10}
    \TSetTick{fa}{3}{11}
    \TSort{(-3,0)}{fa}{4}{l}
    
    \path (-0.8, 1.5) edge[\prio,coopupdate] (-2.2, 1.5);
    \path (4.3, 4.5) edge[\prio,coopupdate] (-2.2, 2.5);
    
    \only<5>{
      \THit{fa_3}{hlv}{c_0}{.east}{c_1}
      \path[bounce, bend right=50]
        \TBounce{c_0}{hlv}{c_1}{.south east};
    }
    
    \only<6->{
      \THit{fa_3}{}{c_0}{.east}{c_1}
      \path[bounce, bend right=50]
        \TBounce{c_0}{}{c_1}{.south east};
    }
  
    \node[labelprio1] at (-1.5,3) {$1$};
    \node[labelprio1] at (-1.5,1.8) {$1$};
    \node[labelprio2] at (-4,3.85) {$2$};
  }
  
  \node[labelprio1] at (1.5,3) {$1$};
  \node[labelprio1] at (1.5,1.8) {$1$};

  \node[labelprio2] at (0,5.4) {$2$};
  \node[labelprio2] at (4.4,3) {$2$};
\end{tikzpicture}
}

\pause[6]
\vspace{.7cm}
\begin{center}
  \f Same dynamics but only 2 priorities
  
  \f Priority \raisebox{-2pt}{\tikz \node[labelprio1] {$1$};} is only for cooperative sorts
\end{center}

\end{frame}

\subsection[x]{Summary}
\input{parts/ccl_prio.tex}

\section{Neutralizing edges}
\subsection[x]{Neutralizing Edges in Process Hitting}
% Arcs neutralisants

\begin{frame}[c]
  \frametitle{Neutralizing Edges}

\begin{columns}
\begin{column}{.4\textwidth}

\begin{tikzpicture}
  %\path[use as bounding box] (-2,0) rectangle (8,6.5);
  \TSort{(0,0)}{a}{2}{l}
  \TSort{(2,0)}{b}{2}{r}
  \TSort{(0,3)}{c}{2}{l}
  \TSort{(2,3)}{d}{2}{r}
  
  \THit{a_0}{}{b_0}{.west}{b_1}
  \path[bounce] \TBounce{b_0}{bend left}{b_1}{.south};
  
  \THit{c_0}{}{d_0}{.west}{d_1}
  \path[bounce] \TBounce{d_0}{bend left}{d_1}{.south};
  
  \node (nea1) at (1,0) {};
  \node[dotne] (nea2) at (1,2.9) {};
  \draw[linene] (nea1) to[out=100, in=-100] (nea2);
  
  \TState{1}{a_0, b_0, c_0, d_0}
  \TState{2}{a_0, b_1, c_0, d_0}
  \TState{3}{a_0, b_1, c_0, d_1}
\end{tikzpicture}

\end{column}
\begin{column}{.55\textwidth}
\begin{center}

\vspace*{1.5cm}
$\PHfrappe{c_0}{d_0}{d_1}$ cannot be played \tval{while}

\bigskip
$\PHfrappe{a_0}{b_0}{b_1}$ is playable

\bigskip
\f Here, only one possible behavior

\end{center}
\end{column}
\end{columns}


\end{frame}

\subsection{Adding Neutralizing Edges to the Metazoan Segmentation}
\input{parts/metazoan_an.tex}
\subsection[x]{Summary}
\input{parts/ccl_an.tex}

\section{Summary \& Conclusion}
% Conclusion

\begin{frame}[c]
  \frametitle{Summary \& Conclusion}

\todo{Point sur les sémantiques}

\todo{Conclusion générale}

\todo{Ouvertures / perspectives}

\todo{Publications}
~

\vfill

Process Hitting: an atomistic modeling with powerful static analysis

\medskip
\begin{enumerate}[1.]
  \item Stochastic parameters:
    \begin{itemize}
      \item To model systems with chronometric features
      \item \tval{Continuous time}
      \item But \tval{hard to analyze}
    \end{itemize}
  \item Classes of priorities:
    \begin{itemize}
      \item Allows to reproduce the same behaviors
      \item Efficient \tval{static analysis}
      \item But the translation to canonical form faces \tval{combinatorial explosion}
    \end{itemize}
  \item Neutralizing edges:
    \begin{itemize}
      \item Alternative to priorities
      \item Closer to reality in some cases
      \item \tval{Lighter translation} to canonical form
    \end{itemize}
\end{enumerate}

\vfill
\Large
\begin{flushright}
  \tval{Thank you}\hspace{1cm}~
\end{flushright}
\vfill

~

\end{frame}


\appendix
\newcounter{finalframe}
\setcounter{finalframe}{\value{framenumber}}

\section[x]{Bibliography}
% Bibliographie

\begin{frame}[c]
  \frametitle{Contributions personnelles}

\small
Chapitre de livre :
\begin{itemize}
  \item Loïc Paulevé, Courtney Chancellor, \emphcolor{Maxime Folschette}, Morgan Magnin et Olivier Roux :
    \emphcolor{Analyzing Large Network Dynamics with Process Hitting},
    \textit{Logical Modeling of Biological Systems},
    éditeurs : Luis Farinas del Cerro et Katsumi Inoue,
    août 2014, ISBN 978-1-84821-680-8.
\end{itemize}

\medskip
Conférences et workshops :
\begin{itemize}
  \item \emphcolor{Maxime Folschette}, Loïc Paulevé, Morgan Magnin et Olivier Roux :
    \emphcolor{Under-approximation of reachability in multivalued asynchronous networks},
    in: Proceedings of the fourth International Workshop on Interactions between Computer Science and Biology, éditeurs : Emanuela Merelli et Angelo Troina,
    \textit{Electronic Notes in Theoretical Computer Science}, \vol 299,
    33--51, Springer Berlin Heidelberg, juin 2013, DOI 10.1016/j.entcs.2013.11.004.
    %\emphcolor{Selected for a special issue in the journal \textit{Theoretical Computer Science}.}
  \item \emphcolor{Maxime Folschette}, Loïc Paulevé, Katsumi Inoue, Morgan Magnin et Olivier Roux :
    \emphcolor{Concretizing the process hitting into biological regulatory networks}, %\newline{}
    in: \textit{Computational Methods in Systems Biology}, éditeurs : David Gilbert et Monika Heiner, %\newline{}
    166--186, Springer Berlin Heidelberg, octobre 2012, DOI 10.1007/978-3-642-33636-2\_11.
  \item \emphcolor{Maxime Folschette}, Loïc Paulevé, Katsumi Inoue, Morgan Magnin et Olivier Roux :
    \emphcolor{Abducing Biological Regulatory Networks from Process Hitting models},
    in: \textit{ECML-PKDD 2012 Workshop on Learning and Discovery in Symbolic Systems Biology}, éditeurs : Oliver Ray et Katsumi Inoue,
    24--35, septembre 2012.
\end{itemize}

\end{frame}



\begin{frame}[c]
  \frametitle{Bibliography}

%\footnotesize
\small
\setlength{\parindent}{-1em}
\setlength{\parskip}{0.5em}

\tcitebullet Loïc Paulevé, Morgan Magnin, Olivier Roux. \ex{Refining dynamics of gene regulatory networks in a stochastic $\pi$-calculus framework}. In Corrado Priami, Ralph-Johan Back, Ion Petre, and Erik de Vink, editors: \textit{Transactions on Computational Systems Biology XIII}, volume 6575 of Lecture Notes in Computer Science, pages 171--191, 2011.

\tcitebullet Loïc Paulevé, Morgan Magnin, Olivier Roux. \ex{Static analysis of biological regulatory networks dynamics using abstract interpretation}. \textit{Mathematical Structures in Computer Science}, 2012.

%\tcitebullet Adrien Richard, Jean-Paul Comet, Gilles Bernot. \ex{R. Thomas' logical method}, 2008. Invited at \textit{Tutorials on modelling methods and tools: Modelling a genetic switch and Metabolic Networks}, Spring School on Modelling Complex Biological Systems in the Context of Genomics.

%\tcitebullet Adrien Richard, Jean-Paul Comet, Gilles Bernot. \textit{Modern Formal Methods and App.}, chapter \ex{Formal Methods for Modeling Biological Regulatory Networks}, pages 83--122, 2006.

%\tcitebullet Maxime Folschette, Loïc Paulevé, Katsumi Inoue, Morgan Magnin, Olivier Roux. \ex{Concretizing the Process Hitting into Biological Regulatory Networks}. In David Gilbert and Monika Heiner, editors, \textit{Computational Methods in Systems Biology X}, Lecture Notes in Computer Science, pages 166--186. Springer Berlin Heidelberg, 2012.

\tcitebullet Maxime Folschette, Loïc Paulevé, Morgan Magnin, Olivier Roux. \ex{Under-approximation of Reachability in Multivalued Asynchronous Networks}. In E. Merelli and A. Troina, editors, \textit{4th International Workshop on Interactions between Computer Science and Biology (CS2Bio’13)}, Electronic Notes in Theoretical Computer Science, Volume 299, 33–51. June 2013.

\tcitebullet Loïc Paulevé. PhD thesis: \ex{\textit{Modélisation, Simulation et Vérification des Grands Réseaux de Régulation Biologique}}, October 2011, Nantes, France.

%\tcitebullet Loïc Paulevé, Morgan Magnin, and Olivier Roux. \textit{Tuning Temporal Features within the Stochastic $\pi$-Calculus}. IEEE Transactions on Software Engineering, 37(6), pages 858--871, 2011.

%\tcitebullet Loïc Paulevé and Adrien Richard. \textit{Topological Fixed Points in Boolean Networks}. Comptes Rendus de l'Académie des Sciences - Series I - Mathematics, 348(15-16), pages 825--828, 2010.

%\tcitebullet Gilles Bernot, Franck Cassez, Jean-Paul Comet, Franck Delaplace, Céline Müller, Olivier Roux. \ex{Semantics of Biological Regulatory Networks}. \textit{Proceedings of the First Workshop on Concurrent Models in Molecular Biology}, Electronic Notes in Theoretical Computer Science 180(3), pages 3--14, 2007.

%\tcitebullet Adrien Richard. \ex{Negative circuits and sustained oscillations in asynchronous automata networks}. \textit{Advances in Applied Mathematics} 44(4), pages 378--392, 2010.

\tcitebullet Paul François, Vincent Hakim, Eric D Siggia. \ex{Deriving structure from evolution : metazoan segmentation}.
In \textit{Molecular Systems Biology}, Volume 3, Issue 1. 2007.

\end{frame}


\section{Adding Priorities}
% Ajout des actions prioritaires pour avoir une équivalence avec les ADN

\begin{frame}[t]
  \frametitle{Adding cooperations}
  \framesubtitle{\tcite{\cpmrtcsb}}

\begin{center}\scalebox{\scaleex}{
\begin{tikzpicture}
  \exphcoop
  
  \TState{9}{a_1,b_1}
  \TState{10}{a_1,b_1,ab_0}
  \TState{11}{a_1,b_1,ab_1}
  \TState{12}{a_1,b_1,ab_3}
  
  \only<9>{
    \node[hl process] at (ab_0.center) {};
    \node[hl process] at (ab_1.center) {};
    \node[hl process] at (ab_2.center) {};
    \node[current process] at (ab_3.center) {};
  }
  
  \only<10>{
    \THit{b_1}{hl}{ab_0}{.210}{ab_1}
    \path[bounce,bend left,hl] \TBounce{ab_0}{}{ab_1}{.240} ;
  }
  
  \only<11>{
    \THit{a_1}{hl}{ab_1}{.160}{ab_3}
    \path[bounce,bend left,hl] \TBounce{ab_1}{}{ab_3}{.south west} ;
  }
\end{tikzpicture}
}\end{center}

\medskip
%\only<-14>{
\begin{liste}
  \item \tval{Cooperation} between \ex{$a_1$} and \ex{$b_1$}: \qex{$\PHfrappe{\underline{a_1 \wedge b_1}}{z_0}{z_1}$}
\pause[5]
  \item Solution: a \tval{cooperative sort} \qex{$ab$} \quad to express \qex{$\underline{a_1 \wedge b_1}$}
\pause[9]
  \item Constraint: each configuration is represented by one process \qex{$\underline{a_1 \wedge b_1} \Rightarrow ab_{11}$}
%\pause[15]
%  \item Advantage: regular sort; drawbacks: complexity, temporal shift
\end{liste}%}
\end{frame}



\begin{frame}[t]
  \frametitle{Adapting the expressivity of PH}
  \framesubtitle{\tcite{\cfpmrcsbio}}

\begin{center}\scalebox{\scaleex}{
\begin{tikzpicture}
  \path[use as bounding box] (-0.5,-0.5) rectangle (6.5,4.5);
  %\path[use as bounding box] (-1,-0.5) rectangle (7.5,5);

  \exphcoopprio{unprio}{}

  \node<5->[process,very thick] at (z_1.center) {?};

  \only<2>{
    \THit{a_1}{selfhit,hlb}{a_1}{.west}{a_0}
    \path[bounce,bend right,hlb] \TBounce{a_1}{}{a_0}{.north west} ;
  }
  \only<3>{
    \THit{b_0.south west}{bend left=90,hlb}{a_0}{.west}{a_1}
    \path[bounce,bend left,hlb] \TBounce{a_0}{}{a_1}{.south west} ;
  }
  \only<4>{
    \THit{b_1}{selfhit,hlb}{b_1}{.west}{b_0}
    \THit{a_0}{bend right=50,hlb}{b_0}{.west}{b_1}
    \path[bounce,bend right,hlb] \TBounce{b_1}{}{b_0}{.north west} ;
    \path[bounce,bend left,hlb] \TBounce{b_0}{}{b_1}{.south west} ;
  }

  \TState{5-6}{a_0, b_0, ab_0, z_0}
  \only<6>{
  \THit{b_0.south west}{hl,bend left=90}{a_0}{.west}{a_1}
  \path[bounce,bend left,hl] \TBounce{a_0}{}{a_1}{.south west} ;
  }
  \TState{7}{a_1, b_0, ab_0, z_0}
  \only<7>{
  \THit{a_1}{hl}{ab_0}{.west}{ab_2}
  \path[bounce,bend left,hl] \TBounce{ab_0}{}{ab_2}{.240} ;
  }
  \TState{8}{a_1, b_0, ab_2, z_0}
  \only<8>{
  \THit{a_1}{selfhit,hl}{a_1}{.west}{a_0}
  \path[bounce,bend right,hl] \TBounce{a_1}{}{a_0}{.north west} ;
  }
  \TState{9}{a_0, b_0, ab_2, z_0}
  \only<9>{
  \THit{a_0}{bend right=50,hl}{b_0}{.west}{b_1}
  \path[bounce,bend left,hl] \TBounce{b_0}{}{b_1}{.south west} ;
  }
  \TState{10}{a_0, b_1, ab_2, z_0}
  \only<10>{
  \THit{b_1}{hl}{ab_2}{.200}{ab_3}
  \path[bounce,bend left,hl] \TBounce{ab_2}{}{ab_3}{.south} ;
  }
  \TState{11}{a_0, b_1, ab_3, z_0}
  \only<11>{
  \THit{ab_3}{hl}{z_0}{.west}{z_1}
  \path[bounce,bend left,hl] \TBounce{z_0}{}{z_1}{.south} ;
  }
  \TState{12-}{a_0, b_1, ab_3, z_1}
\end{tikzpicture}
}\end{center}

\medskip

\tval{Drawback}: Cooperations are too “loose” to be as expressive as ADN.

\medskip

$ \uncover<5->{\PHstate{a_0, b_0, ab_{00}, z_0}}
  \uncover<7->{\rightarrow\PHstate{a_1, b_0, ab_{00}, z_0}}
  \uncover<8->{\rightarrow\PHstate{a_1, b_0, ab_{10}, z_0}}
  \uncover<9->{\rightarrow\PHstate{a_0, b_0, ab_{10}, z_0}}$
\\ \qquad
$ \uncover<10->{\rightarrow\PHstate{a_0, b_1, ab_{10}, z_0}}
  \uncover<11->{\rightarrow\PHstate{a_0, b_1, \redex{ab_{11}}, z_0}}
  \uncover<12->{\rightarrow\PHstate{a_0, b_1, \redex{ab_{11}}, \redex{z_1}}}$

\medskip

\uncover<12->{
The cooperativity should be: \qex{$a_1 \wedge b_1$~\tval{simultaneously}} \quad \textit{i.e.} “in the same state”

\smallskip
but the model behaves like: \qex{$\mathbf{P}(a_1) \wedge \mathbf{P}(b_1)$} \quad with $\mathbf{P}$ = “previously”
}
\end{frame}



\begin{frame}[t]
  \frametitle{Adapting the expressivity of PH}
  \framesubtitle{\tcite{\cfpmrcsbio}}

\begin{center}\scalebox{\scaleex}{
\begin{tikzpicture}
  \path[use as bounding box] (-0.5,-0.5) rectangle (6.5,4.5);

  \exphcoopprio{prio}{}

  \node[process,very thick] at (z_1.center) {?};

  \TState{3}{a_0, b_0, ab_0, z_0}
  \only<3>{
  \THit{b_0.south west}{hl,bend left=90}{a_0}{.west}{a_1}
  \path[bounce,bend left,hl] \TBounce{a_0}{}{a_1}{.south west} ;
  }
  \TState{4}{a_1, b_0, ab_0, z_0}
  \only<4>{
  \THit{a_1}{prio,hl}{ab_0}{.west}{ab_2}
  \path[bounce,bend left,hl] \TBounce{ab_0}{}{ab_2}{.240} ;
  }
  \TState{5}{a_1, b_0, ab_2, z_0}
  \only<5>{
  \THit{a_1}{selfhit,hl}{a_1}{.west}{a_0}
  \path[bounce,bend right,hl] \TBounce{a_1}{}{a_0}{.north west} ;
  }
  \TState{6}{a_0, b_0, ab_2, z_0}
  \only<6>{
  \THit{a_0}{prio,hl}{ab_2}{.160}{ab_0}
  \path[bounce,bend right,hl] \TBounce{ab_2}{}{ab_0}{.north west} ;
  }
  \TState{7}{a_0, b_0, ab_0, z_0}
  \only<7>{
  \THit{a_0}{bend right=50,hl}{b_0}{.west}{b_1}
  \path[bounce,bend left,hl] \TBounce{b_0}{}{b_1}{.south west} ;
  }
  \TState{8}{a_0, b_1, ab_0, z_0}
  \only<8>{
  \THit{b_1}{prio,hl}{ab_0}{.210}{ab_1}
  \path[bounce,bend left] \TBounce{ab_0}{hl}{ab_1}{.240} ;
  }
  \TState{9-}{a_0, b_1, ab_1, z_0}
\end{tikzpicture}
}\end{center}

\begin{itemize}
  \item Prioritise actions updating cooperative sorts (non-biological actions)
  \item All other actions remain unprioritised (evolutions with delays)
\end{itemize}
\pause
$\Rightarrow$ Whenever a regular action is played, all cooperative sorts are already updated

%\medskip
%Now, $z_1$ cannot be reached from $\PHstate{a_0, b_0, ab_{00}, z_0}$

\medskip
$ \uncover<3->{\PHstate{a_0, b_0, ab_{00}, z_0}}
  \uncover<4->{\rightarrow\PHstate{a_1, b_0, ab_{00}, z_0}}
  \uncover<5->{\rightarrow\PHstate{a_1, b_0, ab_{10}, z_0}}
  \uncover<6->{\rightarrow\PHstate{a_0, b_0, ab_{10}, z_0}}$
\\ \qquad
$ \uncover<7->{\rightarrow\PHstate{a_0, b_0, \ex{ab_{00}}, z_0}}
  \uncover<8->{\rightarrow\PHstate{a_0, b_1, \ex{ab_{00}}, z_0}}
  \uncover<9->{\rightarrow\PHstate{a_0, b_1, \ex{ab_{01}}, z_0}}$
\end{frame}



\section{Static analysis with Priorities}
% Analyse statique avec priorités

\begin{frame}[c]
  \frametitle{Static analysis with prioritised actions}
  \framesubtitle{\tcite{\cfpmrcsbio}}

\tval{Sufficient condition:}

\begin{itemize}
  \item no cycle
  \item each objective has a solution
  \item \only<-4>{coherent edges}\only<5->{\sout{coherent edges}}
\end{itemize}
\vspace{1cm}
\hspace{2cm}\uncover<5->{\textcolor{darkyellow}{\textbf{Inconclusive}}}
\vspace{-3cm}

\begin{center}\scalebox{\scaleex}{
\begin{tikzpicture}[aS]
  \priostatic

  \path<2-5> (ab11) edge[aSPrio,Aexedge] (ab11s);
  \path<5-> (ab11) edge[aSPrio,Ahledge] (ab11s);

  \node<3>[Aproc,Aex,at=(a1)] {$a_1$};
  \node<3>[Aproc,Aex,at=(b1)] {$b_1$};
  \node<4->[Aproc,Ahl,at=(a1)] {$a_1$};
  \node<4->[Aproc,Ahl,at=(a0)] {$a_0$};
  %\node<2>[Aproc,Ahl,at=(b0)] {$b_0$};
  %\node<2>[Aproc,Ahl,at=(b1)] {$b_1$};
\end{tikzpicture}
}\end{center}

\scalebox{\scaleminiex}{
\begin{tikzpicture}
  \path[use as bounding box] (-0.5,-0.5) rectangle (8.5,3.5);
  \tikzstyle{current process}=[process,fill=gray]
  \exphcoopprio{prio}{}
  \node[process,very thick] at (z_1.center) {?};
  \TState{1-}{a_0, b_0, ab_0, z_0}
\end{tikzpicture}}
\hfill
\scalebox{\scaleex}{
\scalebox{\scaleex}{
\begin{tikzpicture}[aS]
  \path[use as bounding box] (0,0) rectangle (5.8,4);
  \glclegend{prio}{$z_1$}{$\PHobj{z_0}{z_1}$}
\end{tikzpicture}
}}

\end{frame}




%\input{parts/annex_fixpoint.tex}

%\section{Translating a Process Hitting into a BRN}
%\input{parts/inf.tex}
%\subsection{Interaction Graph Inference}
%\input{parts/inf_ig.tex}
%\subsection{Parametrization Inference}
%\input{parts/inf_param.tex}
%\subsection{Implementation}
%% Implémentation

\begin{frame}[c]
  \frametitle{Traduction en modèle de Thomas}
  \framesubtitle{\tcite{\cfpimrcmsb}}

\begin{itemize}
  \item Inférence du graphe des interactions, puis des paramètres
  \item Analyse exhaustive de la dynamique locale pour chaque régulateur
  \item Possibilité d'énumérer toutes les paramétrisations compatibles avec la dynamique
\end{itemize}

\bigskip
\tval{Complexité} :\\
\quad Linéaire dans le nombre de composants\\ %linear in the number of genes,\\
\quad Exponentielle dans le nombre de régulateurs de chaque composant %exponential in the number of regulators of one gene

\pause
\bigskip
\small
\begin{tabular}{r||c|c|c||c|c||c|c|}
\multicolumn{4}{c||}{Modèles} & \multicolumn{2}{c||}{Inférence du GI} & \multicolumn{2}{c|}{Inférence des paramètres}\\
\hline
\tval{Nom} & Sortes & Processus & Actions & $\Delta t$ & Arcs & $\Delta t$ & Paramètres\\
\hline
  \tval{\ex{egfr20}} & 22 & 152 & 399 & \tval{1s} & 50 & \tval{1s} & 191\\
\hline
  \tval{\ex{tcrsig40}} & 14 & 156 & 301 & \tval{1s} & 54 & \tval{1s} & 143\\
\hline
  \tval{\ex{tcrsig94}} & 39 & 448 & 1124 & \tval{13s} & 169 & $\infty$ & $2.10^9$\\
\hline
  \tval{\ex{egfr104}} & 89 & 748 & 2356 & \tval{4min} & 241 & \tval{1min 30s} & $1.10^6 / 2.10^6$\\
\hline
\end{tabular}

%S = Sortes \quad CS = Sortes coopératives \quad P = Processus \quad A = Actions

\cmodels
\end{frame}

%\subsection{New inference}
%\input{parts/new_inf.tex}

\setcounter{framenumber}{\value{finalframe}}

\end{document}
