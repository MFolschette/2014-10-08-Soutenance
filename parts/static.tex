% Exemples de structures abstraites (graphes de causalité locale)

\begin{frame}
  \frametitle{Static analysis: successive reachability}
  \framesubtitle{\tcite{\cpmrmscs}}

Successive reachability of processes:

\begin{columns}
\begin{column}{0.55\textwidth}

\begin{center}
\scalebox{0.75}{
\begin{tikzpicture}
  %\path[use as bounding box] (-1,-3) rectangle (7,2);
  \exatt

  \TState{1-5}{a_0,b_0,c_0,d_0}
  \TState{6}{a_0,b_0,c_1,d_0}
  \TState{7}{a_0,b_0,c_1,d_1}
  \TState{8}{a_0,b_1,c_1,d_1}
  \node<9>[process,very thick] (d_2) at (d_2.center) {};
  \TState{9}{a_0,b_1,c_1,d_2}

  \node<2>[objective] at (d_1.center) {1?};
  \node<2>[objective] at (d_2.center) {2?};

  \node<3>[objective] at (d_1.center) {1?};
  \node<3>[objective] at (b_1.center) {2?};
  \node<3>[objective] at (d_2.center) {3?};

  \node<4-8>[objective] at (d_2.center) {1?};

  %\node<3>[process,very thick] (d_1) at (d_1.center) {1?};
  %\node<3>[process,very thick] (b_1) at (b_1.center) {2?};
  %\node<3>[process,very thick] (d_2) at (d_2.center) {3?};

  %\node<4-8>[process,very thick] (d_2) at (d_2.center) {1?};

  \only<5>{\THit{a_0}{hlhit}{c_0}{.north}{c_1}}
  \path<5>[bounce,bend left,hlhit] \TBounce{c_0}{}{c_1}{.west};
  \only<6>{\THit{b_0}{hlhit}{d_0}{.west}{d_1}}
  \path<6>[bounce,bend left,hlhit] \TBounce{d_0}{}{d_1}{.south};
  \only<7>{\THit{c_1}{bend left=20pt,hlhit}{b_0}{.west}{b_1}}
  \path<7>[bounce,bend left,hlhit] \TBounce{b_0}{}{b_1}{.south};
  \only<8>{\THit{b_1}{hlhit}{d_1}{.west}{d_2}}
  \path<8>[bounce,bend left,hlhit] \TBounce{d_1}{}{d_2}{.south};
\end{tikzpicture}
}
\end{center}

\end{column}
\begin{column}{0.45\textwidth}

%\pause
~\\~\\
\begin{itemize}
  \item Initial state
    \\ \rex{\PHetat{a_1, b_0, c_0, d_0}} \pause
  \item Objectives
    \\ \rex{$[\ \Rsh d_1\ \PHconcat\ \Rsh d_2\ ]$} \pause
    \\\smallskip \rex{$[\ \Rsh d_1 \PHconcat\ \Rsh b_1 \PHconcat\ \Rsh d_2\ ]$} \pause
    \\\smallskip \rex{$[\ \Rsh d_2\ ]$} \pause
\end{itemize}

\end{column}
\end{columns}

\medskip
\begin{center}
\f Concretization of the objective = scenario

\ex{
$ \only<5>{\underline{\PHfrappe{a_0}{c_0}{c_1}}} \only<-4,6->{\PHfrappe{a_0}{c_0}{c_1}} \PHconcat %
  \only<6>{\underline{\PHfrappe{b_0}{d_0}{d_1}}}\only<-5,7->{\PHfrappe{b_0}{d_0}{d_1}} \PHconcat %
  \only<7>{\underline{\PHfrappe{c_1}{b_0}{b_1}}}\only<-6,8->{\PHfrappe{c_1}{b_0}{b_1}} \PHconcat %
  \only<8>{\underline{\PHfrappe{b_1}{d_1}{d_2}}}\only<-7,9->{\PHfrappe{b_1}{d_1}{d_2}}
$}
\end{center}
\end{frame}



\begin{frame}
  \frametitle{Over- and Under-approximations}
  \framesubtitle{\tcite{\cpmrmscs}}

Static analysis by abstractions:
\begin{fleches}
  \item Directly checking an objective sequence $R$ is hard
  \item Rather check the approximations $P$ and $Q$, where \tval{$P \Rightarrow R \Rightarrow Q$}:
\end{fleches}

\begin{center}
\scalebox{0.6}{
\figsa
}
\end{center}

\only<-7>{~}
\only<8->{
Polynomial w.r.t.~the number of sorts and \\exponential w.r.t.~the number of processes in each sort
\begin{fleches}
  \item Efficient for big models with few levels of expression
\end{fleches}
}
\end{frame}



\begin{frame}
  \frametitle{Under-approximation}

\def \tu {2}
\def \tub {3}
\def \tuf {4}

\begin{columns}
\begin{column}{0.48\textwidth}

\begin{center}
\scalebox{0.55}{
\begin{tikzpicture}
  \exatt
  \TState{-\tu}{a_1,b_1,c_1,d_0}
  \TState{\tub-}{a_0,b_1,c_0,d_0}
  \node[objective] (d_2) at (d_2.center) {?};
\end{tikzpicture}
}
\end{center}

\end{column}
\begin{column}{0.52\textwidth}

\vspace{1.5em}
\tval{Sufficient condition}:

\smallskip
\begin{itemize}
  \item no cycle
  \item \only<-\tu>{each objective has a solution} \only<\tub->{\sout{each objective has a solution}}
\end{itemize}
\begin{center}
  \only<\tu>{\Large\textcolor{darkgreen}{$R$ is \textbf{true}}} \only<\tuf>{\Large\textcolor{darkyellow}{\textbf{Inconclusive}}}
\end{center}

\end{column}
\end{columns}

\begin{center}%
%\vspace*{1cm}%
\scalebox{\scaleex}{%
\only<-\tu>{%
\scalebox{\scaleex}{%
\begin{tikzpicture}[aS]
  \path[use as bounding box] (.7,1) rectangle (5.8,2.5);

  \glclegend{}{$d_2$}{$\PHobj{d_0}{d_2}$}
\end{tikzpicture}
}
  \sauyes
}
\only<\tub->{
  \sauinconc
}}
\end{center}
\end{frame}



\begin{frame}
  \frametitle{Over-approximation}

\def \to {4}
\def \tob {5}
\def \tof {6}
\def \tokp {7}

\begin{columns}
\begin{column}{0.48\textwidth}

\begin{center}
\scalebox{0.55}{
\begin{tikzpicture}
  \exatt
  \TState{-\to}{a_1,b_0,c_0,d_1}
  \TState{\tob-}{a_1,b_1,c_1,d_0}
  \node[objective] (d_2) at (d_2.center) {?};
\end{tikzpicture}
}
\end{center}
\bigskip

\end{column}
\begin{column}{0.52\textwidth}

\tval{Necessary condition}:

\smallskip
\only<2->{
\only<3-\to>{\sout{There exists a traversal}}\only<2,\tob->{There exists a traversal}
with no cycle

\smallskip
\begin{itemize}
  \item \only<3-\to>{\sout{objective $\rightarrow$ follow one solution}}\only<1-2,\tob->{objective $\rightarrow$ follow one solution}
  \item solution $\rightarrow$ follow all processes
  \item process $\rightarrow$ follow all objectives
\end{itemize}
\begin{center}
  \only<\to>{\Large\textcolor{red}{$R$ is \textbf{false}}}\only<\tof->{\Large\textcolor{darkyellow}{\textbf{Inconclusive}}}
\end{center}
}

\end{column}
\end{columns}

\begin{center}
\scalebox{\scaleex}{
\only<1-\to>{
  \saono
}
\only<\tob->{
  \saoinconc
}}
\end{center}
\end{frame}


% 
% \begin{frame}[c]
%   \frametitle{Implementation \& Execution times}
% 
% \Pint\tval{: Existing free OCaml library}
% 
% \medskip
% \f Compiler + tools for Process Hitting models
% 
% \f Documentation \& examples: \lien{http://processhitting.wordpress.com/}
% 
% \pause
% \bigskip
% \medskip
% \tval{Computation time for various reachability analyses:}
% 
% \medskip
% \small
% \begin{tabular}{r||c|c|c|c||c|c|c|}
% \hline
% \tval{Model} & Sorts & Procs & Actions & States & Biocham$^1$ & libddd$^2$ & \Pint \\\hline
% \tval{\ex{egfr20}} & 35 & 196 & 670 & $2^{64}$ & [3s--$\infty$] & [1s--150s] & \tval{0.007s} \\\hline
% \tval{\ex{tcrsig40}} & 54 & 156 & 301 & $2^{73}$ & [1s--$\infty$] & [0.6s--$\infty$] & \tval{0.004s} \\\hline
% \tval{\ex{tcrsig94}} & 133 & 448 & 1124 & $2^{194}$ & $\infty$ & $\infty$ & \tval{0.030s} \\\hline
% \tval{\ex{egfr104}} & 193 & 748 &  2356 & $2^{320}$ &  $\infty$ & $\infty$ & \tval{0.050s}\\\hline
% \end{tabular}
% 
% \medskip
% \quad$^1$ Inria Paris-Rocquencourt/Contraintes\\
% \quad$^2$ LIP6/Move
% 
% \cmodels
% \end{frame}
