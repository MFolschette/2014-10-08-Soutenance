% Analyse statique avec priorités

\begin{frame}[c]
  \frametitle{Analyse statique des Frappes de Processus canoniques}
%  \framesubtitle{\tcite{\cpmrmscs}}

L'ajout de priorités restreint les dynamiques possibles (préemptions)

\smallskip
\f Invalidation de la sous-approximation existante

\begin{center}
\scalebox{0.6}{
\begin{tikzpicture}
  \path[use as bounding box] (-5,-3.5) rectangle (5,3.5);
  \definecolor{r2}{RGB}{238,10,38}

  \draw[shading=2, inner color=r2, outer color=white, rounded corners, draw=none] (-6,3.5) rectangle (6,-3.5);
  \draw[thick,fill=white] (-2.8,2) rectangle (2.8,-2);
  \draw<4->[thick,fill=lightyellow] (-2.8,2) rectangle (2.8,-2);
  \draw<1>[thick, shading=1, top color=darkgreen, bottom color=green,opacity=1] (-1.5,.7) rectangle (1.5,-.7);;
  \draw<2>[thick, shading=1, top color=darkgreen, bottom color=green,opacity=0.3] (-1.5,.7) rectangle (1.5,-.7);;
  \draw<1>[color=darkcyan, ultra thick] (0,0) ellipse (2 and 1.5);
  \draw<2->[color=darkcyan, ultra thick] (0,1.5) arc [start angle=90, end angle=270, x radius=2, y radius=1.5] -- cycle;
  \draw<3->[thick, shading=1, top color=darkgreen, bottom color=green] (-1.5,.7) rectangle (-.3,-.7);;
\end{tikzpicture}
}
\end{center}

\uncover<5->{
Complexité équivalente pour un formalisme plus expressif

\begin{fleches}
  \item Toujours efficace pour de grands modèles
  \item Sous-approximation plus fine
\end{fleches}
}
\end{frame}


\begin{frame}[c]
  \frametitle{Analyse statique des Frappes de Processus canoniques}
  \framesubtitle{\tcite{\cfpmrcsbio}}

\uncover<8->{
\tval{Condition suffisante :}

\begin{itemize}
  \item aucun cycle
  \item tout objectif possède une solution
  \item \only<-11>{cohérence des requis}\only<12->{\sout{cohérence des requis}}
\end{itemize}
\vspace{1cm}
\hspace{2cm}\uncover<12->{\textcolor{darkyellow}{\textbf{Non conclusif}}}
\vspace{-3cm}
}

\begin{center}\scalebox{\scaleex}{
\begin{tikzpicture}[aS]
  \path[draw=none,use as bounding box] (-.5,-2.2) rectangle (12,2.2);
  \node[Aproc] (z1) {$z_1$};
  \uncover<2->{ \node[Aobj,right of=z1] (z01) {$\PHobj{z_0}{z_1}$}; }
  \uncover<3->{ \node[Asol,right of=z01] (z01s) {}; }
  
  \uncover<4->{ \node[Aproc,right of=z01s] (ab11) {$ab_{11}$}; }
  \uncover<5->{ \node[Asol,right of=ab11] (ab11s) {}; }
  
  \uncover<6->{
    \node[Aproc,above right of=ab11s] (a1) {$a_1$};
    \node[Aproc,below right of=ab11s] (b1) {$b_1$};
  }
  
  \uncover<7->{
    \node[Aobj,above right of=a1] (a11) {$\PHobj{a_1}{a_1}$};
    \node[Asol,right of=a11] (a11s) {};
    \node[Aobj,right of=a1] (a01) {$\PHobj{a_0}{a_1}$};
    \node[Asol,right of=a01] (a01s) {};
    \node[Aproc,right of=a01s] (b0) {$b_0$};
    \node[Aobj,right of=b0] (b00) {$\PHobj{b_0}{b_0}$};
    \node[Asol,right of=b00] (b00s) {};
    \node[Aobj,above right of=b0] (b10) {$\PHobj{b_1}{b_0}$};
    \node[Asol,right of=b10] (b10s) {};
    
    \node[Aobj,below right of=b1] (b11) {$\PHobj{b_1}{b_1}$};
    \node[Asol,right of=b11] (b11s) {};
    \node[Aobj,right of=b1] (b01) {$\PHobj{b_0}{b_1}$};
    \node[Asol,right of=b01] (b01s) {};
    \node[Aproc,right of=b01s] (a0) {$a_0$};
    \node[Aobj,right of=a0] (a00) {$\PHobj{a_0}{a_0}$};
    \node[Asol,right of=a00] (a00s) {};
    \node[Aobj,below right of=a0] (a10) {$\PHobj{a_1}{a_0}$};
    \node[Asol,right of=a10] (a10s) {};
  }
  
  \path (z1) edge (z01);
  \path<2-> (z01) edge (z01s);
  \path<3-> (z01s) edge (ab11);
  \path<4-> (ab11) edge[aSPrio] (ab11s);
  \path<5-> (ab11s) edge (a1) edge (b1);
  \path<6-> (a1) edge (a01) edge (a11) (b1) edge (b01) edge (b11);
  
  \path<7->
  (a01) edge (a01s)
  (a01s) edge (b0)
  (a11) edge (a11s)
  (a0) edge (a10) edge (a00)
  (a10) edge (a10s)
  (a00) edge (a00s)
  
  (b0) edge (b10) edge (b00)
  (b10) edge (b10s)
  (b00) edge (b00s)
  (b01) edge (b01s)
  (b01s) edge (a0)
  (b11) edge (b11s)
  ;
  
  % Arc non cohérent
  \node<9-11>[Aproc,Aex,at=(ab11)] {$ab_{11}$};
  \node<9-11>[Asol,Aexsol,right of=ab11] (ab11s) {};
  \path<9-11> (ab11) edge[aSPrio,Aexedge] (ab11s);
  \node<12>[Aproc,Ahl,at=(ab11)] {$ab_{11}$};
  \node<12>[Asol,Ahlsol,right of=ab11] (ab11s) {};
  \path<12> (ab11) edge[aSPrio,Ahledge] (ab11s);
  
  \node<10>[Aproc,Aex,at=(a1)] {$a_1$};
  \node<10>[Aproc,Aex,at=(b1)] {$b_1$};
  \node<11->[Aproc,Ahl,at=(a1)] {$a_1$};
  \node<11->[Aproc,Ahl,at=(a0)] {$a_0$};
\end{tikzpicture}
}\end{center}

\scalebox{\scaleminiex}{
\begin{tikzpicture}
  \path[use as bounding box] (-0.5,-0.5) rectangle (8.5,3.5);
  \tikzstyle{current process}=[process,fill=gray]
  \exphcoopprio{prio}{}
  \node[process,very thick] at (z_1.center) {?};
  \TState{1-}{a_0, b_0, ab_0, z_0}
\end{tikzpicture}}
\hfill
\scalebox{\scaleex}{
\scalebox{\scaleex}{
\begin{tikzpicture}[aS]
  \path[use as bounding box] (0,0) rectangle (5.8,4);
  \glclegend{prio}{$z_1$}{$\PHobj{z_0}{z_1}$}
\end{tikzpicture}
}}

\end{frame}



\begin{frame}[c]
  \frametitle{Implémentation}

Complexité :

\begin{itemize}
  \item Construction du graphe de causalité locale :
  \begin{itemize}
    \item Polynomiale dans le nombre de sortes
    \item Exponentielle dans le nombre de processus de chaque sorte
  \end{itemize}
  \item Analyse du graphe (condition suffisante) :
  \begin{itemize}
    \item Polynomiale dans la taille du graphe
  \end{itemize}
\end{itemize}

\pause
\medskip
L'étude de grands réseaux devient possible :

\bigskip
\small
\begin{tabular}{r||c|c|c|c||c|c|c|}
%\hline
\tval{Modèle} & Sortes & Processus & Actions & États & libddd$^1$ & GINsim$^2$ & \Pint \\\hline
\tval{\ex{egfr20}} & 35 & 196 & 670 & $2^{64}$ & & $<$1s & \tval{0.02s} \\\hline
\tval{\ex{tcrsig40}} & 54 & 156 & 301 & $2^{73}$ & & $\infty$ & \tval{0.02s} \\\hline
\tval{\ex{tcrsig94}} & 133 & 448 & 1124 & $2^{194}$ & [13min -- $\infty$] & & \tval{0.03s} \\\hline
\tval{\ex{egfr104}} & 193 & 748 & 2356 & $2^{320}$ & & & \tval{0.08s}\\\hline
\end{tabular}

\medskip
\quad$^1$ LIP6/Move\\
\quad$^2$ TAGC/IGC

%S = Sorts \quad CS = Cooperative sorts \quad P = Processes \quad A = Actions

%\cmodels
%\medskip
%\todo{Citer les papiers d'origine}
% \citeegfra\\
% \citetcrsiga\\
% \citetcrsigb\\
% \citeegfrb\\
\cmodels

\end{frame}
