% Présentation de l'analyse statique

\begin{frame}
  \frametitle{Approximations de l'analyse d'atteignabilité}
  \framesubtitle{\tcite{\cpmrmscs}}

Vérification des propriétés de la forme :
\begin{center}
  «~Depuis un état initial $s_0$, peut-on atteindre un état $s_n$ où $a_i$ est actif~?~»
\end{center}
Utilisation d'approximations $P$ et $Q$, telles que \tval{$P \Rightarrow R \Rightarrow Q$}

% Static analysis by abstractions:
% \begin{fleches}
%   \item Directly checking an objective sequence $R$ is hard
%   \item Rather check the approximations $P$ and $Q$, where \tval{$P \Rightarrow R \Rightarrow Q$}:
% \end{fleches}

\begin{center}
\scalebox{0.6}{
\begin{tikzpicture}
  \path[use as bounding box] (-5,-3.5) rectangle (5,3.5);
  \definecolor{r2}{RGB}{238,10,38}

  %\path<2->[shading=1, inner color=r2, outer color=white] (3.5,-2.8) -- (4.4,3.2) -- (0,3) -- (-4.5,1.4) -- (-2.5,-2.5) -- (0,-3.6) -- (2.8,-2.8);
  \draw<2->[shading=2, inner color=r2, outer color=white, rounded corners, draw=none] (-6,3.5) rectangle (6,-3.5);
  %\draw<2->[thick,fill=white] (2.5,-2.1) -- (3,2.5) -- (-2.7,1.3) -- (-2,-2) -- (2.5,-2.1);
  \draw<2->[thick,fill=white] (-2.8,2) rectangle (2.8,-2);
  %\draw<6->[thick,fill=lightyellow] (2.5,-2.1) -- (3,2.5) -- (-2.7,1.3) -- (-2,-2) -- (2.5,-2.1);
  \draw<6->[thick,fill=lightyellow] (-2.8,2) rectangle (2.8,-2);

  \node<2->[text width=3.5cm, color=red] (s1) at (-5,2) {Sur-approximation};
  \path<2->[->,very thick,color=red] (s1.south) edge (-3,1.2);
  \node<2->[text width=3cm,color=black] (q) at (2.2,2.3) {$\neg Q$};

  %\draw<4->[thick, shading=1, top color=darkgreen, bottom color=green] (.5,-.8) -- (1,0) -- (.3,1) -- (-1,.5) -- (-.5,-.5) -- (.5,-.8);
  \draw<4->[thick, shading=1, top color=darkgreen, bottom color=green] (-1.5,.7) rectangle (1.5,-.7);;
  \node<4->[text width=3.5cm,color=darkgreen] (s2) at (5.2,-2.5) {Sous-approximation};
  \node<4->[text width=3cm,color=black] (p) at (1.8,.4) {$P$};
  \path<4->[->,very thick,color=darkgreen] (s2) edge (1,-.8);

  \node[text width=3cm,align=center,color=darkcyan] (s) at (0,-1.7) {Solution exacte};
  \node<1->[text width=3cm,color=darkcyan] (s0) at (0,0) {};
  \draw[color=darkcyan, ultra thick] (0,0) ellipse (2 and 1.5);
  \node[text width=3cm,color=black] (r) at (2,1.2) {$R$};

  \only<3->{
    \node[point] at (-3.3,-1) {};
    \node[point] at (2,2.5) {};
  }
  \only<5->{
    \node[point] at (-1,.2) {};
    \node[point] at (1.1,-.5) {};
  }
  \only<7->{
    \node[point] at (-.5,-1.1) {};
    \node[point] at (2.5,1) {};
  }
\end{tikzpicture}
}
\end{center}

\uncover<8->{
Polynomial dans le nombre de sortes\\
Exponentiel dans le nombre de processus de chaque sorte
\begin{fleches}
  \item Efficace pour de grands modèles avec peu de niveaux d'expression
\end{fleches}
}
\end{frame}
