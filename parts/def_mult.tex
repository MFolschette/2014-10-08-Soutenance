% Actions plurielles

\begin{frame}[c]
  \frametitle{Actions plurielles}

\begin{columns}
\begin{column}{.4\textwidth}

\begin{tikzpicture}
  %\path[use as bounding box] (-2,0) rectangle (8,6.5);
  \TSort{(0,0)}{b}{2}{l}
  \TSort{(0,3)}{a}{2}{l}
  \TSort{(3,0)}{d}{2}{r}
  \TSort{(3,3)}{c}{2}{r}
  
  \TActionPlur{a_0, b_1}{c_0.west}{c_1.south west}{}{1.5,1.5}{left}
  \TActionPlur{}{d_0.west}{d_1.south west}{}{1.5,1.5}{left}
  \TActionPlur{}{c_1.east}{c_0.north east}{}{4,5}{left}

  \TState{1}{a_0, b_1, c_0, d_0}
  \TState{2}{a_0, b_1, c_1, d_1}
  \TState{3}{a_0, b_1, c_0, d_1}
\end{tikzpicture}

\end{column}
\begin{column}{.55\textwidth}
\begin{center}

\begin{itemize}
  \item Synchronisations entre les actions :
  \begin{itemize}
    \item[--] Présence conjointe de réactifs
    \item[--] Consommation simultanée d'éléments
    \item[--] Production simultanée
  \end{itemize}
  \item Représentation d'équations biochimiques :
\end{itemize}

\smallskip
$A + B + C \rightarrow C + D$

\vspace*{.5cm}

\ex{$h_1 = \PHfrappemults{c_1}{c_0}$
$h_2 = \PHfrappemults{a_1, b_1, c_0, d_0}{c_1, d_1}$}

\vspace*{.5cm}
$\PHfrappemult{A}{B}$

est jouable dans $s$ si et seulement si :

$A \subset s$

\bigskip
Après jeu : $s \play (\PHfrappemult{A}{B}) = s \Cap B$

\end{center}
\end{column}
\end{columns}


\end{frame}
