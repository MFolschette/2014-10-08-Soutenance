% Actions plurielles

\begin{frame}[c]
  \frametitle{Frappes de Processus avec actions plurielles}

\begin{tikzpicture}
  \path[use as bounding box] (-5.2,-4) rectangle (5.2,3.5);
  \planPHstandard
  \planPHp
  \planPHan
  \planPHmult
  \planPHcanonique[stillhidden]
\end{tikzpicture}

\end{frame}



\begin{frame}[c]
  \frametitle{Introduction d'actions plurielles}

\begin{columns}
\begin{column}{.4\textwidth}

\begin{tikzpicture}
  %\path[use as bounding box] (-2,0) rectangle (8,6.5);
  %\TSort{(0,0)}{b}{2}{l}
  \TSort{(0,2)}{y}{2}{l}
  \TSort{(3,0)}{z}{2}{r}
  \TSort{(3,3)}{x}{2}{r}
  
  \TActionPlur{y_1}{x_1.west}{x_0.north west}{}{1.5,2.5}{right}
  \TActionPlur{}{z_0.west}{z_1.south west}{}{1.5,2.5}{left}
  \TActionPlur{}{x_0.east}{x_1.south east}{}{4,2.5}{right}

  \TState{1}{x_1, y_1, z_0}
  \TState{2}{x_0, y_1, z_1}
  \TState{3}{x_1, y_1, z_1}
\end{tikzpicture}

\end{column}
\begin{column}{.55\textwidth}
\begin{center}

\begin{itemize}
  \item Synchronisations entre les actions :
  \begin{itemize}
    \item[--] Présence conjointe de réactifs
    \item[--] Consommation simultanée d'éléments
    \item[--] Production simultanée
  \end{itemize}
  \item Représentation d'équations biochimiques :\\
    \centering $X \xrightarrow{Y} Z$\\
    \raggedright sous la forme :\\
    \centering $h_2 = \PHfrappemults{x_1, y_1, z_0}{x_0, z_1}$\\
\end{itemize}

% \vspace*{.5cm}

% \ex{%
% $h_2 = \PHfrappemults{x_1, y_1, z_0}{x_0, z_1}$\\
% $h_1 = \PHfrappemults{c_0}{c_1}$%
% }

\vspace*{.5cm}
Tous les processus de $A$\\
doivent être présents pour jouer $\PHfrappemult{A}{B}$

\medskip
Après le jeu de $\PHfrappemult{A}{B}$,\\
tous les processus de $B$ sont présents
% 
% est jouable dans $s$ si et seulement si :
% 
% $A \subset s$
% 
% \bigskip
% Après jeu : $s \play (\PHfrappemult{A}{B}) = s \Cap B$

\end{center}
\end{column}
\end{columns}


\end{frame}
