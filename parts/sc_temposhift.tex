% Ajout des actions prioritaires pour avoir une équivalence avec les ADN

\begin{frame}[c]
  \frametitle{Frappes de Processus canoniques}

\begin{tikzpicture}
  \path[use as bounding box] (-5.2,-4) rectangle (5.2,3.5);
  \planPHstandard
  \planPHp
  \planPHan
  \planPHmult
  \planPHcanonique
\end{tikzpicture}

\end{frame}



\begin{frame}[t]
  \frametitle{Décalage temporel des sortes coopératives}
  \framesubtitle{\tcite{\cfpmrcsbio}}

\begin{center}\scalebox{\scaleex}{
\begin{tikzpicture}
  \path[use as bounding box] (-0.5,-0.5) rectangle (6.5,4.5);
  %\path[use as bounding box] (-1,-0.5) rectangle (7.5,5);

  \exphcoopprio{unprio}{}

  \node<5->[process,very thick] at (z_1.center) {?};

  \only<2>{
    \THit{a_1}{selfhit,hlb}{a_1}{.west}{a_0}
    \path[bounce,bend right,hlb] \TBounce{a_1}{}{a_0}{.north west} ;
  }
  \only<3>{
    \THit{b_0.south west}{bend left=90,hlb}{a_0}{.west}{a_1}
    \path[bounce,bend left,hlb] \TBounce{a_0}{}{a_1}{.south west} ;
  }
  \only<4>{
    \THit{b_1}{selfhit,hlb}{b_1}{.west}{b_0}
    \THit{a_0}{bend right=50,hlb}{b_0}{.west}{b_1}
    \path[bounce,bend right,hlb] \TBounce{b_1}{}{b_0}{.north west} ;
    \path[bounce,bend left,hlb] \TBounce{b_0}{}{b_1}{.south west} ;
  }

  \TState{5-6}{a_0, b_0, ab_0, z_0}
  \only<6>{
  \THit{b_0.south west}{hl,bend left=90}{a_0}{.west}{a_1}
  \path[bounce,bend left,hl] \TBounce{a_0}{}{a_1}{.south west} ;
  }
  \TState{7}{a_1, b_0, ab_0, z_0}
  \only<7>{
  \THit{a_1}{hl}{ab_0}{.west}{ab_2}
  \path[bounce,bend left,hl] \TBounce{ab_0}{}{ab_2}{.240} ;
  }
  \TState{8}{a_1, b_0, ab_2, z_0}
  \only<8>{
  \THit{a_1}{selfhit,hl}{a_1}{.west}{a_0}
  \path[bounce,bend right,hl] \TBounce{a_1}{}{a_0}{.north west} ;
  }
  \TState{9}{a_0, b_0, ab_2, z_0}
  \only<9>{
  \THit{a_0}{bend right=50,hl}{b_0}{.west}{b_1}
  \path[bounce,bend left,hl] \TBounce{b_0}{}{b_1}{.south west} ;
  }
  \TState{10}{a_0, b_1, ab_2, z_0}
  \only<10>{
  \THit{b_1}{hl}{ab_2}{.200}{ab_3}
  \path[bounce,bend left,hl] \TBounce{ab_2}{}{ab_3}{.south} ;
  }
  \TState{11}{a_0, b_1, ab_3, z_0}
  \only<11>{
  \THit{ab_3}{hl}{z_0}{.west}{z_1}
  \path[bounce,bend left,hl] \TBounce{z_0}{}{z_1}{.south} ;
  }
  \TState{12-}{a_0, b_1, ab_3, z_1}
\end{tikzpicture}
}\end{center}

\medskip

\tval{Inconvénient} : les coopérations sont trop «~lâches~» (décalage temporel)

\medskip

$ \uncover<5->{\PHstate{a_0, b_0, ab_{00}, z_0}}
  \uncover<7->{\rightarrow\PHstate{a_1, b_0, ab_{00}, z_0}}
  \uncover<8->{\rightarrow\PHstate{a_1, b_0, ab_{10}, z_0}}
  \uncover<9->{\rightarrow\PHstate{a_0, b_0, ab_{10}, z_0}}$
\\ \qquad
$ \uncover<10->{\rightarrow\PHstate{a_0, b_1, ab_{10}, z_0}}
  \uncover<11->{\rightarrow\PHstate{a_0, b_1, \redex{ab_{11}}, z_0}}
  \uncover<12->{\rightarrow\PHstate{a_0, b_1, \redex{ab_{11}}, \redex{z_1}}}$

\medskip

\uncover<12->{
\begin{tabular}{ll}
  Comportement attendu : & \ex{$a_1 \wedge b_1$~\tval{simultanément}} \quad \cad «~dans le même état~»
\\
  Comportement obtenu : & \ex{$\mathbf{P}(a_1) \wedge \mathbf{P}(b_1)$} \quad avec $\mathbf{P}$ = «~précédemment~»
\end{tabular}
% Comportement attendu : \qex{$a_1 \wedge b_1$~\tval{simultanément}} \quad \cad «~dans le même état~»
% 
% \smallskip
% Comportement obtenu : \qex{$\mathbf{P}(a_1) \wedge \mathbf{P}(b_1)$} \quad avec $\mathbf{P}$ = «~précédemment~»
}
\end{frame}
