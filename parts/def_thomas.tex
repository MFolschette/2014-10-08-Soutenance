% Définition du Réseaux Discrets Asynchrones

\newcommand{\Fadn}{\mathbb{F}}
\newcommand{\Eadn}{\mathbb{E}}
\newcommand{\SGadn}{\mathrm{G}}

\begin{frame}[c]
  \frametitle{The Asynchronous Discrete Networks (ADN)}
  \framesubtitle{\todo{Citer DeJong}}

\begin{itemize}
  \item A set of components \qex{$N = \{ a, b, z \}$}
\uncover<2->{
  \item A set of expression levels for each component \qex{$z \in \Fadn^z = \segm{0}{2}$}
  \item The set of global states \qex{$\Fadn = \Fadn^a \times \Fadn^b \times \Fadn^z$}
}
\uncover<3->{
  \item An evolution function for each component \qex{$f^z : \Fadn \rightarrow \Fadn^z$}
\end{itemize}

\begin{center}
\begin{tabular}{ccc}
  \ex{$f^a = \neg b$} & \ex{$f^b = b \vee \neg a$} & \ex{$f^z = a + b$} \vspace{.5em}\\
  \begin{tabular}[t]{c|c}
    $b$ & $f^a(b)$ \\
  \hline
    $0$ & $\mathbf{1}$ \\
    $1$ & $\mathbf{0}$ \\
  \end{tabular}
&
  \begin{tabular}[t]{cc|c}
    $a$ & $b$ & $f^b(a, b)$ \\
  \hline
    $0$ & $0$ & $\mathbf{1}$ \\
    $0$ & $1$ & $\mathbf{1}$ \\
    $1$ & $0$ & $\mathbf{0}$ \\
    $1$ & $1$ & $\mathbf{1}$
  \end{tabular}
&
  \begin{tabular}[t]{cc|c}
    $a$ & $b$ & $f^z(a, b)$ \\
  \hline
    $0$ & $0$ & $\mathbf{0}$ \\
    $0$ & $1$ & $\mathbf{1}$ \\
    $1$ & $0$ & $\mathbf{1}$ \\
    $1$ & $1$ & $\mathbf{2}$
  \end{tabular}
\end{tabular}
}

\bigskip

\begin{tikzpicture}[adn]
  \path[use as bounding box] (-0.7,-0.7) rectangle (2.5,2);
  \node[inner sep=0] (z) at (2,0.75) {z};
  \node[inner sep=0] (a) at (0,1.5) {a};
  \node[inner sep=0] (b) at (0,0) {b};
  \path<2->
    node[elabel, above=-1em of a] {$\segm{0}{1}$}
    node[elabel, below=-1em of b] {$\segm{0}{1}$}
    node[elabel, below=-1em of z] {$\segm{0}{2}$};
  \path<3->
    (a) edge[bend right=15] (b)
    (b) edge[bend right=15] (a)
    (b) edge[loop left] (b)
    (a) edge (z)
    (b) edge (z);
\end{tikzpicture}
\end{center}
\end{frame}



\begin{frame}[c]
  \frametitle{The Asynchronous Discrete Networks (ADN)}

State Graph: $\SGadn = (\Fadn, \Eadn)$, where one component evolves at a time given its function $f^a$
$$(x, y) \in \Eadn \Longleftrightarrow \exists a \in N, y^a = f^a(x) \wedge \forall b \neq a, y^b = x^b$$

%\f Only one component at a time $a$ evolves to reach the value of its evolution function $f^a(x)$

\pause
\medskip
%\begin{tabular*}{\textwidth}{@{\extracolsep{\fill}}lcc}
%  Size of the State Graph: & $\displaystyle|\Fadn| = \prod_{a \in N} |\Fadn^a| \quad\geq 2^{|N|}$ &
%\end{tabular*}
Size of the State Graph: \quad $\displaystyle|\Fadn| = \prod_{a \in N} |\Fadn^a| \quad\geq 2^{|N|}$

\medskip
\f \tval{Exponential} in the number $|N|$ of components

\pause
\bigskip
Some works give a link between the structure and the behaviour of an ADN
\begin{itemize}
  \item \tval{Thomas' conjecture} (condition for multiple fixed points or attractive cycle)
  \begin{itemize}
    \item Boolean: \tcite{\todo{citeremy}}
    \item Multivalued: \tcite{\todo{citerichardcomet}}
  \end{itemize}
\end{itemize}

\pause
\medskip
But methods related to reachability rely on the State Graph

e.g.: \ex{Starting from $(a, b, z) = (0, 0, 0)$, can the system reach $z = 2$ ?}
\begin{itemize}
  \item \tval{Temporal logics}
  \begin{itemize}
    \item CTL: \tcite{\todo{citesmbionet}}
    \item LTL: \tcite{\todo{citeito}}
  \end{itemize}
\end{itemize}
\end{frame}
