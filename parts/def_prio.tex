% Définition des priorités

\begin{frame}[c]
  \frametitle{Introduction of Classes of Priorities}
  \framesubtitle{\tcite{\cfpmrcsbio}}

\begin{itemize}
  \item Each action is associated to a class of priority.
  \item An action cannot be played if another action of higher priority is playable.
\end{itemize}

\medskip

\begin{center}
\begin{tabular}{ccccc}
  \tikz \node[labelprio1] {$1$}; &
  \tikz \node[labelprio2] {$2$}; &
  \tikz \node[labelprio3] {$3$}; &
  \vspace*{.5em}\hspace*{.3cm}\ldots\hspace*{.3cm} &
    \tikz \node[labelprio, fill=gray!40] {$n$}; \\\hline
  \parbox{1cm}{\vspace*{.5em}highest priority} &
  \parbox{1cm}{~} & \parbox{1cm}{~} &&
  \parbox{1cm}{\vspace*{.5em}lowest priority}
\end{tabular}
\hspace*{-1em}
\raisebox{2.2pt}{$\blacktriangleright$}

\bigskip
\bigskip

\pause

\begin{tikzpicture}
  \path[use as bounding box] (-0.5,-0.5) rectangle (2.5,1.5);
  \TSort{(0,0)}{a}{2}{l}
  \TSort{(2,0)}{b}{2}{r}
  \THit{a_0}{}{b_0}{.west}{b_1}
  \THit{a_0}{out=-120,in=180,selfhit}{a_0}{.west}{a_1}
  \path[bounce]
  \TBounce{a_0}{bend left}{a_1}{.south}
  \TBounce{b_0}{bend left}{b_1}{.south}
  ;
  \TState{-2}{a_0,b_0}
  \TState{3-}{a_1,b_0}

  \node[labelprio1] at (-1.5,-0.5) {$1$};
  \node[labelprio2] at (1,0.25) {$2$};
\end{tikzpicture}

\bigskip

\f $b_1$ \tval{cannot be reached}
\end{center}

\end{frame}
