% Définition des priorités

\begin{frame}[c]
  \frametitle{Frappes de Processus avec classes de priorités}

\begin{tikzpicture}
  \path[use as bounding box] (-5.2,-4) rectangle (5.2,3.5);
  \planPHstandard
  \planPHp
  \planPHan[stillhidden]
  \planPHmult[stillhidden]
  \planPHcanonique[stillhidden]
\end{tikzpicture}

\end{frame}



\begin{frame}[t]
  \frametitle{Introduction de classes de priorités}
  \framesubtitle{\tcite{\cfpmrcsbio}}

\bigskip
\begin{itemize}
  \item À chaque action est associée une classe de priorité
  \item Une action n'est jouable que si aucune action plus prioritaire ne l'est
\end{itemize}

\medskip

\begin{center}
% \begin{tabular}{ccccc}
%   \hspace*{.3cm}\tikz \node[labelprio1] {$1$}; \hspace*{.3cm} &
%   \hspace*{.3cm}\tikz \node[labelprio2] {$2$}; \hspace*{.3cm} &
%   \hspace*{.3cm}\tikz \node[labelprio3] {$3$}; \hspace*{.3cm} &
%   \vspace*{.5em}\hspace*{.3cm}\raisebox{5pt}{\ldots}\hspace*{.3cm} &
%   \hspace*{.3cm}\tikz \node[labelprion] {$n$}; \hspace*{.3cm} \\\hline
%   \multicolumn{2}{l}{
%   \parbox{1.5cm}{\vspace*{.5em}plus haute\\priorité}} &
%   %\parbox{1cm}{~} &
%   \parbox{1cm}{~} &&
%   \parbox{1.5cm}{\vspace*{.5em}plus basse\\priorité}
% \end{tabular}
% \hspace*{-1em}
% \raisebox{2.2pt}{$\blacktriangleright$}

\begin{tabular}{*{5}{>{\centering}p{1cm}}}
  \tikz \node[labelprio1] {$1$}; &
  \tikz \node[labelprio2] {$2$}; &
  \tikz \node[labelprio3] {$3$}; &
  \raisebox{5pt}{\ldots} &
  \tikz \node[labelprion] {$n$};
\vspace*{.5em} \tabularnewline \hline
  \multicolumn{2}{l}{\parbox{1.5cm}{\vspace*{.5em}le plus\\prioritaire}} &&
  \multicolumn{2}{r}{\parbox{1.5cm}{\raggedleft\vspace*{.5em}le moins\\prioritaire}}
\end{tabular}
\hspace*{-1em}
\raisebox{2.2pt}{$\blacktriangleright$}

\bigskip

\only<2-3>{
\bigskip
\begin{tikzpicture}
  \path[use as bounding box] (-0.5,-0.5) rectangle (2.5,1.5);
  \TSort{(0,0)}{a}{2}{l}
  \TSort{(2,0)}{b}{2}{r}
  \THit{a_0}{}{b_0}{.west}{b_1}
  \THit{a_0}{out=-120,in=180,selfhit}{a_0}{.west}{a_1}
  \path[bounce]
  \TBounce{a_0}{bend left}{a_1}{.south}
  \TBounce{b_0}{bend left}{b_1}{.south}
  ;
  \TState{-2}{a_0,b_0}
  \TState{3-}{a_1,b_0}

  \node[labelprio1] at (-1.5,-0.5) {$1$};
  \node[labelprio2] at (1,0.25) {$2$};
\end{tikzpicture}

\bigskip

\f $b_1$ n'est \tval{jamais atteignable}
}
\end{center}

\only<4->{
\begin{itemize}
  \item Permet de modéliser des classes d'actions de vitesses similaires
\end{itemize}
\begin{center}
\begin{tabular}{*{5}{>{\centering}p{1cm}}}
  \tikz \node[labelprio1,labelstocha] {$A$}; &
  \tikz \node[labelprio2,labelstocha] {$B$}; &
  \tikz \node[labelprio3,labelstocha] {$C$}; &
  \raisebox{5pt}{\ldots} &
  \tikz \node[labelprion,labelstocha] {$N$};
\vspace*{.5em} \tabularnewline \hline
  \multicolumn{1}{r}{\parbox{1cm}{\hspace*{-1.7cm}\parbox{2.5cm}{\raggedleft\vspace*{.5em}\tval{instantanée}\\(non contrôlable)}}} &
  \multicolumn{2}{l}{\parbox{2cm}{\vspace*{.5em}\tval{très rapide}\\(contrôlable)}} &
  \multicolumn{2}{r}{\parbox{2cm}{\raggedleft\vspace*{.5em}\tval{très lente}\\~}}
\end{tabular}
\hspace*{-1em}
\raisebox{2.2pt}{$\blacktriangleright$}
\end{center}
}

\end{frame}
