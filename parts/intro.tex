% Intro

\begin{frame}[c]
  \frametitle{Une double problématique}

La modélisation d'un système est la première étape vers sa compréhension

\begin{center}
\begin{tikzpicture}
  \node[ellipse, fill=blue!20] (m) at (-1.5, 0) {Modélisation};
  \node[ellipse, fill=violet!20] (a) at (1.5, 0) {Analyse};
  \uncover<2->{ \path[->, shorten >=1em, shorten <=1em] (a) edge[ultra thick, bend left] (m); }
  \uncover<3->{ \path[->, shorten >=1em, shorten <=1em] (m) edge[ultra thick, bend left] (a); }
\end{tikzpicture}
\end{center}

\pause[2]
L'analyse recherchée impacte les choix de modélisation
\begin{itemize}
  \item Les outils de modélisation doivent être adaptés aux propriétés observées
\end{itemize}

\pause[3]
\medskip
Les choix de modélisation impactent les résultats de l'analyse
\begin{itemize}
  \item Un modèle trop grossier donne peu d'informations
  \item Un modèle de grande taille augmente le temps d'analyse
\end{itemize}

\pause[4]
\medskip
\begin{center}
\tval{Les étapes de modélisation et d'analyse d'un système sont indissociables}
\end{center}

\end{frame}



\begin{frame}[c]
\frametitle{Plan de la présentation}

\tval{État de l'art} de la modélisation des réseaux de régulation biologique
\begin{itemize}
  \item Modélisations discrètes asynchrones et modèle de Thomas
  \item Frappes de Processus standards
\end{itemize}

\pause
\bigskip
\tval{Enrichissement} de la modélisation des Frappes de Processus
\begin{itemize}
  \item Intégration de données ou de contraintes temporelles
  \item Éléments de Synchronisation entre les actions
  \item[] \quad \f Ajout de priorités, d'arcs neutralisants ou d'actions plurielles
\end{itemize}

\pause
\bigskip
\tval{Analyse} des Frappes de Processus
\begin{itemize}
  \item Correction des sortes coopératives
  \item Analyse statique pour l'atteignabilité
  \item Équivalences avec d'autres formalismes
\end{itemize}

\end{frame}
