
\begin{frame}[c]
  \frametitle{Forme canonique de la segmentation des métazoaires}

\makenoprio

\vspace*{.5cm}
\scalebox{.9}{
\begin{tikzpicture}
  \path[use as bounding box] (-5.75,0) rectangle (5.75,5.5);
  \TSort{(-5,4)}{c}{2}{l}
  \TSort{(0,1)}{f}{2}{l}
  \TSort{(5,4)}{a}{2}{r}

  \TSetTick{fc}{0}{00}
  \TSetTick{fc}{1}{01}
  \TSetTick{fc}{2}{10}
  \TSetTick{fc}{3}{11}
  \TSort{(3,0)}{fc}{4}{r}
  
  \THit{fc_2}{}{a_0}{.south west}{a_1}
  \path[bounce, bend left=60]
    \TBounce{a_0}{}{a_1}{.south west};
  
  \THit{c_1.north east}{}{a_1}{.west}{a_0}
  \path[bounce, bend right=50]
    \TBounce{a_1}{}{a_0}{.north west};
  
  \path (0.8, 1.5) edge[\prio,coopupdate] (2.2, 1.5);
  \path (-4.3, 4.5) edge[\prio,coopupdate] (2.2, 2.5);
  
  \only<1>{
    \THit{c_1.north}{selfhit}{c_1}{.west}{c_0}
    \path[bounce, bend right=50]
      \TBounce{c_1}{}{c_0}{.north west};
  }
  
  \only<2>{
    \THit{c_1.north}{selfhit,hlr}{c_1}{.west}{c_0}
    \path[bounce, bend right=50]
      \TBounce{c_1}{hlr}{c_0}{.north west};
  }
  
  \only<-2>{
    \node[labelprio3] at (-4.4,6) {$3$};
  }
  
  \only<3>{
    \THit{a_0.west}{hlv}{c_1}{.east}{c_0}
    \path[bounce, bend left=50]
      \TBounce{c_1}{hlv}{c_0}{.north east};
    \node[labelprio2] at (0.5,4.7) {$2$};
  }
  
  \only<4->{
    \THit{a_0.west}{}{c_1}{.east}{c_0}
    \path[bounce, bend left=50]
      \TBounce{c_1}{}{c_0}{.north east};
    \node[labelprio2] at (0.5,4.7) {$2$};
  }
  
  \only<-4>{
    \node[labelprio3] at (-3,2.3) {$3$};
  }
  
  \only<-3>{
    \THit{f_1}{bend left=30, in=90}{c_0}{.west}{c_1}
    \path[bounce, bend left=50]
      \TBounce{c_0}{}{c_1}{.south west};
  }
  
  \only<4>{
    \THit{f_1}{bend left=30, in=90, hlr}{c_0}{.west}{c_1}
    \path[bounce, bend left=50]
      \TBounce{c_0}{hlr}{c_1}{.south west};
  }
  
  \only<5->{
    \TSetTick{fa}{0}{00}
    \TSetTick{fa}{1}{01}
    \TSetTick{fa}{2}{10}
    \TSetTick{fa}{3}{11}
    \TSort{(-3,0)}{fa}{4}{l}
    
    \path (-0.8, 1.5) edge[\prio,coopupdate] (-2.2, 1.5);
    \path (4.3, 4.5) edge[\prio,coopupdate] (-2.2, 2.5);
    
    \only<5>{
      \THit{fa_3}{hlv}{c_0}{.east}{c_1}
      \path[bounce, bend right=50]
        \TBounce{c_0}{hlv}{c_1}{.south east};
    }
    
    \only<6->{
      \THit{fa_3}{}{c_0}{.east}{c_1}
      \path[bounce, bend right=50]
        \TBounce{c_0}{}{c_1}{.south east};
    }
  
    \node[labelprio1] at (-1.5,3) {$1$};
    \node[labelprio1] at (-1.5,1.8) {$1$};
    \node[labelprio2] at (-4,3.85) {$2$};
  }
  
  \node[labelprio1] at (1.5,3) {$1$};
  \node[labelprio1] at (1.5,1.8) {$1$};

  \node[labelprio2] at (0,5.4) {$2$};
  \node[labelprio2] at (4.4,3) {$2$};
\end{tikzpicture}
}

\pause[6]
\vspace{.7cm}
\begin{center}
  \f Même dynamique aux sortes coopératives supplémentaires près
  
  \medskip
  \f Il est possible de calculer la forme canonique\\pour toutes les extensions des Frappes de Processus
\end{center}

\end{frame}
