% Arcs neutralisants

\begin{frame}[c]
  \frametitle{Frappes de Processus avec arcs neutralisants}

\begin{tikzpicture}
  \path[use as bounding box] (-5.2,-4) rectangle (5.2,3.5);
  \planPHstandard
  \planPHp
  \planPHan
  \planPHmult[stillhidden]
  \planPHcanonique[stillhidden]
\end{tikzpicture}

\end{frame}



\begin{frame}[c]
  \frametitle{Introduction d'arcs neutralisants}

\begin{columns}
\begin{column}{.4\textwidth}

\begin{tikzpicture}
  %\path[use as bounding box] (-2,0) rectangle (8,6.5);
  \TSort{(0,0)}{a}{2}{l}
  \TSort{(2,0)}{b}{2}{r}
  \TSort{(0,3)}{c}{2}{l}
  \TSort{(2,3)}{d}{2}{r}
  
  \THit{a_0}{}{b_0}{.west}{b_1}
  \path[bounce] \TBounce{b_0}{bend left}{b_1}{.south};
  
  \THit{c_0}{}{d_0}{.west}{d_1}
  \path[bounce] \TBounce{d_0}{bend left}{d_1}{.south};
  
  \node (nea1) at (1,0) {};
  \node[dotne] (nea2) at (1,2.9) {};
  \draw[linene] (nea1) to[out=100, in=-100] (nea2);
  
  \TState{1}{a_0, b_0, c_0, d_0}
  \TState{2}{a_0, b_1, c_0, d_0}
  \TState{3}{a_0, b_1, c_0, d_1}
\end{tikzpicture}

\end{column}
\begin{column}{.55\textwidth}
\begin{center}

\begin{itemize}
  \item Intégration de données temporelles\\
    concernant les temps de réaction relatifs
  \item Préemptions atomiques entre les actions\\
    similaires à des «~priorités atomiques~»
\end{itemize}

\vspace*{1cm}
$\PHfrappe{c_0}{d_0}{d_1}$ ne peut être jouée \tval{tant que}

\bigskip
$\PHfrappe{a_0}{b_0}{b_1}$ est jouable

\bigskip
\f $d_1$ est \tval{toujours} atteint après $b_1$

\end{center}
\end{column}
\end{columns}


\end{frame}
