% Actions plurielles sur Metazoan

\begin{frame}[t]
  \frametitle{Segmentation métazoaire avec actions plurielles}

\makenoprio

\begin{tikzpicture}[apdotsimple/.style={apdot}]
  \path[use as bounding box] (-3,1) rectangle (6,6.5);
  \TSort{(0,4)}{c}{2}{l}
  \TSort{(1,0)}{f}{2}{l}
  \TSort{(5,4)}{a}{2}{r}

  \TActionPlur{f_1, c_0}{a_0.west}{a_1.south west}{}{2.5,2.5}{left}
%   \THit{fc_2}{\prio}{a_0}{.west}{a_1}
%   \path[bounce, bend left=50]
%     \TBounce{a_0}{\prio}{a_1}{.south west};
  
  \THit{c_1}{\prio}{a_1}{.west}{a_0}
  \path[bounce, bend right=50]
    \TBounce{a_1}{\prio}{a_0}{.north west};
  
  \THit{f_1}{bend left=30, in=90}{c_0}{.west}{c_1}
  \path[bounce, bend left=50]
    \TBounce{c_0}{}{c_1}{.south west};
  
  \TActionPlur{}{c_1.west}{c_0.north west}{}{-1,6}{right}
%   \THit{c_1}{selfhit}{c_1}{.west}{c_0}
%   \path[bounce, bend right=50]
%     \TBounce{c_1}{}{c_0}{.north west};
\end{tikzpicture}

\begin{flushright}
  \f Même dynamique qu'avec des priorités,\\
  à la sorte coopérative absente près.
\end{flushright}

\end{frame}
